\newpage

\scan{102}

\newpage

% Translation below

% så see till att du kommer med din klinga innan till / aldramest i
% hans halfwa styrkia. Så snart han då åter will under din Klinga
% cavera, moste du gifwa gran acht uppå i dett han går igenom / att di
% stöter tert' tillijka med honom utan till öfwer hans halfwa styrkia
% till hans högre bröst / med ett insteg af högre foten.

ensure that your blade is inside / mostly in his half strong. The
moment he seeks to go under your Blade, you must pay great attention
for when he goes through / that you thrust third outside over his half
strong to his right chest, with a step in of the right foot.

\exercise{}

% Ligger din wederpart med en lös Klinga fram för dig / så sök att du
% kommer utan till hans Klinga / att du åst öfwer din högre Arm
% aldeles betäckt; Wille han då innan till öfwer din Klinga inkasta en
% qvart', så gif wäl acht uppå hans tempo. i dett han begynner att
% röra sin hand och Klinga/ att du sätter i dett samma tempo din
% wenstre foot fram/ och stöter qvartam innan till åht hans högre
% bröst: Men du moste föra din högre hand och fäste wäl högt/ der med
% att hufwudet blifwer des bättre betäckt. Fig. 9.

If your opponent is before you with a loose Blade / then seek to come
outside his Blade / that you are over your right Arm fully covered; If
he then inside over your Blade throw a fourth, then pay good attention
to his tempo. As he begins to move his hand and Blade, that you in the
same tempo put your left foot forward, and thrust fourth inside to his
right chest. But you must move your right hand and hilt well high, so
that your head is better covered. Fig. 9.

\exercise{}

% Har du string' din Fiende innan till / så gif grant i det han wille
% lyfta sin Klinga öfwer din / at du i ett ögneblick förwender din hand
% i secunda, och sänk kroppen wehl / paisera under hans Kjlinga till
% hans högra sijda bort. Fig. 22. Du kant och i dett han kastar öfwer
% din Klinga / sättia din högra fot bak om di wenstra tilbaka / så
% långt du förmår / förwend din hand i secunda, din högra Arm streck
% wähl ut / med din wenstra hand parerar du hans Klinga till din högra
% sijda sinkos.

If you have bound your enemy on the inside, then pay great attention
when he seeks to lift his Blade above yours / that you in the blink of
an eye twist your hand into second, and lower your body well, pass
under his Blade to his right side. Fig. 22. You can also when he
throws over your Blade, put your right foot back behind the left, as
far as you are capable / twist your hand into second, your right Arm
extend well, with your left hand parry his Blade low to your right
side.

% \exercise{Huruledes Stringering och Ligering emot dhe medel- och underläger blifwa brukade.}

\exercise{How various binds are used against the middle and low positions}

% Ligger din Advers' med jämbn Klinga i medel tert' med sin udd intet
% högre än sitt [wekalijff ???] / så string' honom innan till med
% förhängder udd; Ligger han stilla / så gack i den enga mens' och låt
% i dett samma ditt kors öfwer hans Klinga nedersiunka / din udd förer
% du så mycket öfwer dig / wille han ändå intet movera sig / så stöt
% halfwa qvart med din Klinga öfwer hans swaga / med ophögder udd till
% hans högre Axel / och ett insteg af högre foten: Förskränk dig i
% qvart' med mera säkerheet af wenstre handen / så och utantill per
% tertiam mutatis mutandis.

If your opponent is with a horizontal BLade in middle third with his
point no higher than his [belly]\sidenote{Original text is hard to
make out, but is plausibly ``wekalijff'', which is an archaich term
for the soft part of the torso between hop bone and lower edge of the
rib cage, so belly would be an accurate translation}, then bind him on
the inside with a hanging point. If he staus still, then go to short
measure and let your cross sink over his Blade, your point brought up
as much above you, if he still does not move, thust half fourth with
your Blade over his weak, with elevated point to his right Shoulder
and a step in with the right foot. Lean in fourth with more security
from the left hand, and so outside in third, with appropiate changes.

\exercise{}

% Ligger han som tillförende; Så string' honom innan till / wille han
% intet cavera, battera med din styrkia i hans swaga; Doch blif med
% fötterne ferm, stöt honom med en half Qvart' ofwan till hans högre
% Axel a. pie fermo eller passera utan till när battuten är giord /
% wäl till hans högre sijda sinkos.

If he is like before: Bind him on the inside, if he does not
disengage, beat with your strong in his weak. Nonetheless, remain with
ur feet form, thrust him with a half fourth above to his right
Shoulder a pie fermo or pass outside when the beat is done, well bent
to his right side .

\exercise{}

% Ähr hans udd uti ett Medel-läger / något opphögd / så string' honom
% med ophögder udd / så mycket / att du ifrå den linien, som går ifrån
% hans udd till din kropp / åst

If is point is in a Middle position, womewhat elevated, then bind him
with an elevated point, so much, that you from the line, that goes
from his point to your body
