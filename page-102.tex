\newpage

\scan{102}

\newpage

% Translation below

% så see till att du kommer med din klinga innan till / aldramest i
% hans halfwa styrkia. Så snart han då åter will under din Klinga
% cavera, moste du gifwa gran acht uppå i dett han går igenom / att di
% stöter tert' tillijka med honom utan till öfwer hans halfwa styrkia
% till hans högre bröst / med ett insteg af högre foten.

\exercise{}

% Ligger din wederpart med en lös Klinga fram för dig / så sök att du
% kommer utan till hans Klinga / att du åst öfwer din högre Arm
% aldeles betäckt; Wille han då innan til löfwer din Klinga infalla en
% qvart', så gif wäl acht uppå hans tempo. i dett han begynner att
% röra sin hand och Klinga/ att du sätter i dett samma tempo din
% wenstre foot fram/ och stöter qvartam innan till åht hans högre
% bröst: Men du moste föra din högre hand och fäste wäl högt/ der med
% att hufwudet blifwer des bättre betäckt. Fig. 9.

\exercise{}

% Har du string' dn iFiende innan till / så gif grant i det han wille lyfta sin Klinga öfwer din / at du ett ögneblick förwender din hand i secuna, och sänk kroppen wehl / paisera under hans Kjlinga till hans högra sijda bort. Fig. 22. Du kant och i dett han kastar öfwer din Klinga / sättia din högra fot bak om di wenstra tilbaka / så lågt du förmår / förwend din hand i secunda, din högra Arm streck wähl ut / med din wenstra hand parerar du has Klinga till din högra sijda sinkos.

% \exercise{Huruledes Stringering och Logering emot dhe medel- och underläger blifwa brukade.}

% Ligger din Advers' med jämbn Klinga i medel tert' med sing udd intet
% högre än sitt [wekaltiff ???] / så string' honom innan till med
% sörhängder udd; Ligger han stilla / så gack i den enga mens' och låt
% i dett samma ditt kors öfwer hans Klinga nefersiunka / din udd förer
% du så mycket öfwer dig / wille han ändå intet movera sig / så stöt
% halfwa qvart med din Klinga öfwer hans swaga / med ophögder udd till
% hans högre Axel / och ett insteg af högre foten: Förskränk dig i
% qvart' med mera säkerheet af wenstre handen / så och utantill per
% tertiam mutatis mutandis.

\exercise{}

% Ligger han som tillförende; Så string' honom innan till / wille han
% intet cavera, battera med din styrkia i hans swaga; Doch blif med
% fötterne ferm, stöt honom med en half Qvart' ofwan till hans högre
% Axel a. pie fermo eller passera utan till när battuten är giord /
% wäl till hans högre sijda sinkos.

\exercise{}

% Ähr hans udd uti ett Medel-läger / något opphögd / så string' honom
% med ophögder udd / så mycket / att du ifrå den linien, sm går ifrån
% hans udd till din kropp / åst
