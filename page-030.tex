\newpage

\scan{030}

\newpage

% Translation below
% Fötterna går i dett samma mer unierat. hwilken samandrechtigheet
% förorsakar en särdeles styrka och geschwindigheet i sielfwa
% förrättandet/ och kan man i ingången en effect i den andra
% beqwemligen förändra / altså att Fienden med besfär kan sig
% defendera; Blifwer honom och tijden i månge ränker att gi"ora, eller
% af dens andras förtagande rätt att judicera betagen / och oasedt
% occasion löper snart förbij / och när man har passerat hans udd är
% man för hans stöt säker.

% Men i stöten a pie fermo händer sig offta / att man offta trädt s/a
% diupt in / eller och Fienden i dett samma har gåt fram för sig / har
% han råkat så diupt in i mens' att man intet har kunnat komma
% tillbaka / och uti retireingen blifwit stötter ; I hwilket fall är
% gott / att man går fort in till sin Fiendes kropp / emedan den
% största faran är när man kommer till mens' men när udden är
% passerat, och man förföllier till hans kropp / kommer man till honom
% för än han kan retirera sin Klinga.

% Dett händer sig att man har med udden passerat, och stöt att Fienden
% drager Armen i det samma tillbaka och laederar ; En dett är dens
% faute som passerade, att han intet Continuerade till sin Fiendes
% kropp eller intet tagit tempo wäl; En om han i dett samma / när
% Fienden förer Klingan fram för sig / eller i dett han är med deffeta
% occuperat eller går utur presenza, kan han intet i den samma tijden
% / när man passerar på sådant satt retorera eller dra tillbaka.

% Är och nödvändigt när man passerar att man aktijd föllier Fiendens
% Klinga / och altijd blifwer hos den samma att man säkert der hos
% fortgå kan.

% Men det äro några andra / när de allredan aldeles hafwer passeeat,
% sig retirera och söka att stöta / hwilket lättigare låter sig giöra
% med en stackot än med en lång Klinga; der emot man måtte wetta / att
% Klingan hon må wara lång eller stackot / så lär dock den samme som
% passerar, går och wäl sutten / och då han skulle komma med Klingan
% så diupt med en scurso ingå till sin Fiendes kropp; En uti passering
% kan man giöra åtskillige ting / först med kroppe att man stöter sin
% Fiende / och honom der med disordonerar, sedan taga i hans fäste /
% så kan man och om man intet har råkat med passaden, sättia sin
% foot bak om sin Fiendes och med ringning kasta honom till marken.

% I lijka måtto är den samma som passerar, i all tillfällen mycket
% ferdigare och beredsammare än den andra som är med diffesa
% occuperat, och uti den faram son han beinner sig uti blifwer
% confunderat.

% Den som wäl weet och kan passera, förer Klingan med mera qvis terta,
% twingar sin Fiende mycket bättre / är mycket säkrare på sin sak /
% särdeles när han 
