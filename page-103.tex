\newpage

\scan{103}

\newpage

% Translation below

% det försäkrat / och stöter med korset i den högden af hans idd wähl
% betäckt / med ophögder udd till hans högre Axel.

\exercise{}

% Ligger han åter uti ett medel- eller underläger / med jämdn Klinga /
% så string' honom innan till / Caverar han med jämbd Klingar eller
% med en string' utan avandement, så Cavera tillijka med honom, och
% ryck i dett samma något bättre i mens' Caverar han åter igen / så
% stöt utan per tertiam med ophögder udd / och ditt kors öfwer hans
% swaga med passaden, eller a. p. fermo till hans högre Axel.

\exercise{}

% Stringerar din Advers' dig med ett insteg / så kan du med
% Contra-cavation stöta / och bruka innan till dett falska steget:
% Märk här jämpte / att du innan till stringerar på dhe medel- och
% underläger / förandes ditt kors wähl till sin högra sijda / på dett
% att du icke där öppnar dig / caverar han under din uhtwertes stöt /
% så förwerka utur tert' i qvarta. Du kant och giöra honom en half
% stöt utan till med din styrkia i hans swaga / der han då i dett
% samma skulle under din stöt cavera, och stöta en Qvart till ditt
% öfwerlijff / så gif acht i det han stöter / att du förwänder din
% hand i secunda, och förfaller med din kropp under hans Klinga
% Fig. 21. Wijdare / har din Fiende stringerat din Klinga utan till /
% så stå med din Kropp något högt / och gack i det samma med din
% Klinga under hans igenom / och giör en half stöt med din styrkia i
% hans swaga / caverar din contrapart i dett samma / och wille stöta
% öfwer din Klinga äht din högra Axel / så cavera / och förwend din
% hand i secunda, och stöt till hans högra bröst ( din kropp sänker du
% wäl Fig. 21.

\exercise{}

% Ett annat manier. Ligger han i ett medel- eller underläger / eller
% muterar der uti / så låt din klinga jämpt under sig siunka; Ditt
% kors holt fast effter ditt Ansichte

