\newpage

\scan{017}

\newpage

% Translation below

% jämpt och wäl gå till din Fiendes högra sida / der medd han intet
% utan till under din klinga kan instöta / det ware sig at du dersädes
% giorde en Chiamat, och han i det samma stötte / du med quarta togo
% innan till contratempo. I det öfrige måste det samma taga i acht /
% hwad tillförende hoos contrapostur är omtalt och påmint; Om du nu på
% det sättet di Fiendes Klinga genom sogget[???} string hafwer giordt
% henne underkastat / hwilket är en begynnelse till victorie, och weet
% at behålla din Klinka frij / l'ar han intet kunna stöta dig / utan
% sig snarare retirera än approchera, så länge till des han seer sitt
% advantage. I fall han med sin Udd i en continuerlig rörelse wore /
% eller för dig i en halff eller heel Circel med stadig arm muterade,
% och du hans swaga ofwan till intet wäl kunde stringera / utan
% der eöfwer med din Klinga kunde blifwa disordonnerar[???] och
% stötter / skall du i ställe / när hans klinga st/ar till hans högra
% ut / med udden innan tll jämpt till hans kors som det centrum, der
% hans kling sing minst rörrer tillgå; men förer han Klingn i mutering
% till sin wänstra / s/a gack utan till i en jämn linea til hans kors
% / rörer han klingan tillijka med Armen / så gack ti hans högre Axel;
% Din udd blifwer s8/a w"al utan som innan til ( när du på det s8'att
% tiltr8"ader hans kors eller axel) förd uthi en halff qvarta,
% undantagandes när han med sin udd ligger lägre än ditt Bälte / d/a
% skeer det i Tert. på hwilket sätt han blifwer twungen att liggia
% stilla med sin klinga / eller kommer då så diupt in / at du honom i
% en s/adan mouvement stöta kan / och han sedan effter sin lngsamme
% förwäntn intet kan parera. Men emoot Secunda och Prima, går du under
% hans klinga i en jämn linie per Quartam till hans kors; Skulle han
% nu med sin klinga ligga stilla / kan du honom den samme ofwan til
% string. eller der han intet wille liggia stilla / kan du utan til
% ifrå hans sida giöra en ligade. Ehuruwäl den rörelsen men en hel
% Circel skeer / kant du på fölliande sätt ofwan til wäl
% förstängia. Giff acht uppå til hwilken sijda circulen ofawn til
% omkringgår, Går han ofwan effter ifrån sin h¦ogra till sin Wenstra /
% så gack med din udd jämpt till hans kors såsom de centro, då låter
% han ingen gång röra dun kilnga. Ophäff med en hast din udd / i det
% hans klinga kommer ofwan effter omkring / och wändt honom innan
% till, så har du funnit hans klinga innan til, stringera altså; men
% gieck hans klinga ofwan til ifrån hans Wenstra till hans Högre / så
% uphäf din udd / at din klinga blifwer något uthwärtes wänder / då
% hafwer du henne der med stringerat. Giff altså wijdare acht uppå din
% Adversarii motioner, hwar effter du ditt arbete kan rätta och
% bestyra.

\chap{On tempo and contratempo}

% Tempo kallas här den rörelsen / som din Adversarius giör uti
% mensura och dig igenom den samma oundwijkeligen / emedan han intet
% kan giöra twå rörelser på en gång / så att du får en ypning till att
% stöta / heller och elliest en occasion derigenom winner / som är dig
% till en fördel; Lijkwäl skall den motion att stöta / intet wara
% större och längre än tempo är / som dig blifwer gifwit utaf din
% Adversarius. En elliest på sådant sätt att han fådt tiid att parera
% / förr än du har hindt honom / och kunnat satt [ fatt ??? ] dig
% imedlertijd uti fahra. Der emot om du blifwer warje din Fiendes
% rörelse / l"arer din stöt utan twifwel lyckas.