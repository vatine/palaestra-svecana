\newpage

\scan{109}

\newpage

% Translation below

% stöt med secund wäl betäckt. Fig. 21. Så kan du och när du hafwer
% innan till string. och han stöter utan till öfwer din Klinga / så
% förwänt din hand i prima. Fig. 6. Stöter han utan till i din halfwa
% styrkia / så voltera qvarta utan till öfwer hans högre Arm till hans
% högre sijda / dock icke att hans Klinga röres. Fig. 15.

thrust second well covered. Fig. 21. You can also when you have bound
inside and he thrusts outside over your Blade, turn your hand into
prime. Fig. 6. If he thrusts outside in your half strong, then turn
your blade outside over his right Arm to his right side, but do not
touch his blade. Fig. 15.

\exercise{}

% Stöter han i din styrkia / så gack tillijka med honom och stöt utan
% till Contratempo; Stöter han innan eller utan till åht din högra
% sijda angulerat / så bruka ligaden der emot / och stöt henne uti ett
% tempo utan till per tert' med försänkter udd / tillijka med dett
% falska steget till hans inwersz kropp eller låt ditt kors effter
% siunka / och stöt utan till öfwer hans Klinga per Tert'. Detta måste
% du såsom en General avertiment, i all din fechtning observera, der
% med du weet när du skall eller intet skall Cavera, ty när du gifwit
% din Fiende mycket jern / kan du icke utan stort tempo (såframt då
% intet retirerar dig med en heel cavation) stöta / derföre moste du
% bruka en half cavation. Den andra är / när du hafwer gifwit lijtet
% järn / så skeer dett med mindre motion; Men det måste der hos wara
% hans motion av fötterne / när du wilt på hans string' stöta. Du går
% i contratempo intet högre med ditt kors än som hans swaga är / och
% blifwer du altijd när hans swaga / men går han med sin swaga / eller
% udd helt ur presens så behåller du din jämna linie i stöten.

If he thrusts in your strong, then go with him and thrust outside
Countertempo. If he thrusts in- or out-side angled to your right side,
use the bind against it, and thrust within one tempo outside in third
with lowered point, also with the false step to his inside body, or
let your cross sink, and thrust outside over his Blade in third. This
you must as a general advertisment, in all your fencing observe, that
you know when to or when not to Disengage, since when you have given
your Enemy much iron, you cannot without a large tempo (unless you
retreat with a whole disengage) thrust, therefor you must use a half
disengage. The second is, when you have given small iron, it happens
with smaller movement. But there must be his motion of the feet, when
you want to thrust on his bind. You never go higher in countertemp
with your cross than his weak, and always be by his weak, but if he
goes with his weak, or point fully out of presence you keep an even
line in the thrust.

\exercise{}

% Stringera honom innan till / och giör honom en fint derstädes per
% quart med contrapostur, wille han i det samma med en angulerad
% Secunda innan till ditt öfwerlijff stöta / så giör en
% contra-angulum, och passera med Secunda innan till hoos hans Klinga
% fort; Eller förfall med din kropp under hans Klinga / lijtet till
% hans högra sinkos / och wändt tillijka med honom

Bind him inside, and do him a feint in fourth with conuterpose, if he
then thrusts an angled Second inside to your upper body, then do him a
counter-angle, and pass with Second inside by his Blade quickly. Or
fall with you bodu under his BLade, somewhat low and to his right, and
turn with him
