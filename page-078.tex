\newpage

\scan{078}

\newpage

% Translation below

% \exercise{Emot Mutering,}
\exercise{Against Mutation}

% Ligger han i Tert'. med jembd Klinga och Muterar för dig / så att
% hans kropp är innan till blott / då gack icke effer hans swaga /
% utan gack innan till i prospective hans kors: när du är i Mens'
% eller han träder i dett samma / så stöt du då men försänktier udd.
% / Qvart' till hans underlijff / så är hans Klinga der med ligerat;
% eller stöt Qvart' avancerat till hans öwerlijff. Eller Gvardera dig
% med ett medel-läger / i dett då han med mutering avancerar, så gack
% utan till med finten per Tert'. uti Contrapostur opförande / så att
% du funnit hans Klinga / Eller gack i Qvart' när du hafwer nått Mens'
% i prospective hans kors under hans Klinga / så att du kommer utan
% till hans Klinga / och stöt det på Qvart' innan till / med omwreden
% och försänkter udd. Fig- 7.

\exercise{}
% Gack honom utan till i prospective hans kors/ Cavera derpå med en
% hast under hans Klinga/ så att du då kommer innan till öfwer hans
% Klinga/ och avancera i dett samma med den högre foten/ stöt der på i
% ett arbete Terz hos hans Klinga omwriden utan till under hans Klinga
% med försänkter udd Fig. 8. Du kan rätt och slätt ända fram när du
% äst honom angången i prospective till hans kårs/ och när du äst i
% Mens' och han går med sin udd till sin wänstra / stöt Terz' utan
% till ända fram öfwer hans högra arm.

\exercise
% Ett annat emot mutering; Gå honom utan till hans högre öfwer-lijff
% så långe med din udd/ i prospective hans kårs/ till des du kommer
% med din Klinga atraverio öfwer hans/ så gjör dig hans Mutering ingen
% skada / stöt i dett samma Tertz utan till fort; skulle han/det samma
% innan til willia Voltera så fått din Klinga baksiunka/ och för din
% wänstre hand fram med ditt kårs/ då kan han intet bruka Qvart': Du
% kant och giöra Finter war igenom du bringar honom att liggia stilla
% / och tag sedan Mens' och tempo i acht.

\exercise{}
% Muterar han och går med sin udd en heel Circle omkring / så gack
% honom emot i en jembd linea med din udd i Qvarta, till hans kårs /
% då lähr han intet röra din Klinga / gif gran oacht uppå hwilken
% sijda Circelen okring löper; ty går han ifrå sin högra till sin
% wänstra / så stöt effter undefängen Mens', i dett han kommer med sin
% Klinga neder utan till, med uddens försänkning i ett arbete Qvart'
% lijkasom wriden innan till hans högre bröst. Fig. 7.

\exercise{}
% Eller ophäf din udd i dett samma han kommer ofwan omkring/ så
% blifwer han förhindrat i sin mutering, och hans Klinga innan till
% String' wille han då din Klinga innan till forcera, så Cedera i
% samma tempo med din Klinga / och stöt med angulerat secunda innan
% till hans öfwerlijff af hans kropp Fig. 10. Eller behålt udden som
% tillförende emot hans kårs: stöt i dett du har fått Mens' (nähr han
% går med kårset neder till son högra) med secund' emot hans Klinga
% till hans inwersz kropp.

\exercise{}
% Muterar han och går med sin Klinga i en heel Circel ifrå hans
% wenstra till hans högra/ så uphäf din udd i dett han kommer ofwan
% effter omkring / så är hans Klinga stängd; wille han forcera din
% Klinga / så kom med ditt kårs i ett tempo innan till öfwer hans
% Klinga i qvart. falzera der med hans Klinga / och stöt med samma
% qvart innan till åht hans högre höfft / eller voltera qvart' utan
% till öfwer hans högre arm. Fig. 7.
