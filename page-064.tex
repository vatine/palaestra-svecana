\newpage

\scan{064}

\newpage

% Translation below
