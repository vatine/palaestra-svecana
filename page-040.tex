\newpage

\scan{040}

\newpage

% Translation below
% Men sådant måste skee med en lijten retirade, särdeles när Fienden
% är kommen temmeligen när i mensuram; En skulle du i dett samma
% avancera, blefwo du utan all twifel råkat och stötter.

But this must happen with a small retreat, especially when the Enemy
has come pretty close in Measure. If you were to advance in the same,
you would without doubt be reached and thrusted.

% Är fördenskull en sådan retirade god / när du dig för en förestående
% fara wilt liberera, så länge till des du får tempo att stöta / och om
% den Wederpart intet achtade det, utan jempt fort stötte / kan du i
% dett samma (emädan du står ferms med dine fötter) under hans stöt
% med en hast gå igenom till en annan sijda. Du kant och wäl i
% retirering / om du wilt förändra ditt läger / och således förtaga
% hans förehafwande och förhindra hans dessein [???]

Such a retreat is also good, when you against a present danger wants
to [liberera???]\sidenote{Defend?}, long enough that you get tempo to
thrust / and if your Opponent does not pay attention, but quickly
thrusts you can in the same (since you stand firm with your feet)
under his thrust quickly go through to another side. You can also in
retreating, if you want to change your position, and thus preempt his
action and stop his [dessein???]\sidenote{The original Swedish is a
bit unclear, but that's what it looks like and I don't have a
translation at the moment}.

% I lijka måtto när du får tillfälle att string' hans Klinga / och han
% skulle gå henne der emot / sökandes att Forcera henne / kan du med
% willia wijka undan med din Klinga / och låta fara hans Klinga / så
% lär han utan twifwel giöra en Cadure, i hwilken tijd du lärer förr
% råka honom än han kommer tillbaka till din Klinga / eller elliest
% får någon avantage öfwer henne.

In equal measure when you have an opportunity to bind his Blade / and
he should go against it / seeking to Force it / you can willingly veer your
Blade aside / and let go of his blade / he will without doubt make a
Cadure, in which time you should sooner reach him than he comes back
to your Blade / or at least get an advantage over it.

% Och är detta dett säkraste/ hwar till behöfwes mindre force, än man
% skulle sättia sig honom emot: Men lått i dett samma din udd som
% tillförede siunka / faller han henne då effter kan du råka honom
% öfwer hans Klinga; Man kan och på många andra sätt befrija sin
% Klinga / att han intet finner henne / förr än uti stöten / och så
% mycket mindre / emedan du redan har fått string' hans Klinga / och
% den ene emot den andra formerat är: Håller han emot din Klinga i
% larga mensura och gifwer sig under blott, måtte du tillförende /
% förr än du stöter/ med Cavation på Fintesätt öfwer till hans Klinga
% gå bättre i mensuram / der med han intet kan hålla udden fram för
% dig / och på sådant sätt twinga honom att gifwa dig tempo.

And this is the safest / to which you need less force, than should you
go against him. But let in the same your point sink / if he falls
after it you can reach him over his Blade. You can also in many other
ways free your Blade, that he does not find it, until in the thrust,
and so much less, since you have already bound his Blade, and one
against the other is formed. If he holds against your blade in wide
measure and give an opening on the underside, you must, before you
thrust, with a disengage in Feint-manner over his Blade, go to better
measure, thus he cannot hold the point in front of you, and in such a
manner force him to give you tempo.

% \chap{Huruledes du skall förhålla dig emot Finter.}
\chap{How you should relate yourself against Feints}

% De giöra somblige Finterna mehr med foten än med Klingan / i dett att
% de så hårdt som dem möjeliget är battera deras Fiende eller
% Wederpart / der med att sträckia och skräma honom  /och uti sådant
% skrämande söka att stöta.

Some make feints more with the foot than with the Blade / in that they
as hard as they can beat their Enemy or Opponent / thus to stretch and
scare him, and in such scaring seek to thrust.

% Detta låter sig giöra uti en Sahl / der man ibland en tillrörelsen
% dubitera kunde; Men på blotta marken der intet något liud gifwes lär
% intet

This can be done in the Salle, where one sometimes could doubt the
movement. But on plain ground where no sound is made, would not
