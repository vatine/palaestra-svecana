\newpage

\scan{058}

\newpage

% Translation below

% högre Axel / at han griper med sin styrka effter din swaga / och
% geer sig i det samma utan til öfwer sin högre Arm blotter / så
% caverar du i samma [???] och söker Ansichtet komma innan til din
% Klinga / på det sättet kan man altijd stöta qvarta utan till /
% allenast at kroppen har natur tertia / stöt altså per qvartam eller
% tertium öfwer hans högre Arm in. Och så när han effter din inwertes
% fint utan jern parerar; Men ginge han med en strak impetu utur
% presenza behöfwes intet at du föllier med korset till hans swaga/
% utan stöt långt med jämbd Klinga öfwer fort.
right Shoulder, that he reaches with his string after your weak, and
give himself in this  outside and above his arm an opening, so you
disengage and seek to place your Face inside your blade, in this
manner you can always thrust fourth on the outside, since the body has
a natural third, thrust thus in fourth or third over hist Arm in. Thus
when he after your inwards feint parry without iron; But if he went
with a strong imposition out of presence it is not needed to follow
with the cross to his weak, but thrust far with an even blade quickly.

\chap{The Thrust in Fouth}

% Det äro fyra manier till at stöta qvarta. Den första är  a" p" f"
% med halfwa kroppen / och den högre Foten jempt för sig fram uht / i
% denna qvarta / skall du intet som somblige giöra / hålla Hufwudet
% tillbaka / utan häldre upphöja fästet något till din diffesa /
% särdeles emot secund; Men i det / at fästet med högre hande
% kommer så högt op / måtte man med Ansichtet nederböjt åht
% Armen /
There are four manners of thrusting in fourth. The first is a pie
fermo with half the body and the right foot straight out ahead of you,
in this fourth, you should not as some do, hold your Head back, but
rather raise your hilt to your defence, especially against second. But
in that the hilt with the right hand ome so high, you must with the
Face somewhat bent to the Arm,
%          att du på det sättet seer din Fiende bak om ditt fäste
% allenast med höger ögat; fördenskul blifwer och Armen innan til
% något angulerat det med du på hans werkning kan taga bättre achta /
% och således parera hans Klinga ifrån dig/ hwilket så wäl a: p: f:
% som i den {\it girerte} och pasierande i qwarta wil nödvändigtvis
% wara observerat;
that you in this manner can see your Enemy behind your hilt with your
right eye; in this manner your Arm is also slightly angled  and you
can on his workings can better act, and thus parry his Blade away from
you, which both in a: p: f: which in the rotating and passing in
fourth will by necessity be observed.
%                  Den ófre delen af din kropp skal du tillijka dit
% Ansichte jämpt fram uträckia / ty om du wille hålla Hufwudet
% tillbaka / skulle det intet wara så säkert / erföre / ju wijdare den
% samma ifrån styrkan af Klinga / ju mehr och mehr blifwer hon öppen
% och blotter / såsom och kan du en håller på sådant sät din stöt
% förlängia / at du til din Fiendes kropp hinner; 
The upper part of your body you should, like your Face, always extend
forward, since if you were to hold your Head back, it would not be as
safe, since, the wider the same is from the strong of the Blade, the
more it is open and uncovered, also you can with this extend your
thust, that you to your Enemy's body get.
%                                                  Här hoss wijdare
% skal du i acht taga/ det du stöten i {\it qvarta} inter förwder /
% förr n du har fåt {\it mensuram} / och först . udden / sedan
% Hufwudet / handen tillijka korset gieck til has swaga/ så öpnar du
% dig intet så bittijd i den högre sijdan / en heller låter hans udd
% ofwan til {\it angulum} mindre lufft/ 
In this you should further take into account, that you in the thrust
in fourth not turn, until youhave gained measure, and first the point,
then the Head, the hand and cross goes to his weak, thus you do not
open yourself early on the right side, and you deny his point above
angled less air,
%                                       hwilket emot de undergvard
% särdeles will tagas i acht / och om du kommer med din stöt tillijka
% med Klingan i {\it qvarta} diupare in . och effter {\it proportion}
% måtte du din kropp mehr och mer halbera; En i det at stöten har
% råkat / så af Klingan och kropper tillijka fulkomlig har wändt sig/
% hwilket med en hast uti ett {\it tempo} i det Foten sättes in. Emot
% de medel och undergvard blifwer {\it qvarta} öfwer hans Klinga med
% försänkt korss / och upphögd udd / til hans högre Axel stötter
% allena at kroppen blifwer wänder.
which against the underguard should be observed, and if you come with
your thrust with your Blade in fourth even deeper, you must in
proportion your body more and more bend. Until in the thrust has
landed, the Blade and the body have completely turned, which in a
hurry in one tempo, in that the Foot is placed. Against the middle and
lower guard the fourth is thrust over hos Blade with lowered guard,
and elevated point, to his right Shoulder with only the body being turned.
% Den andre arten at stöta per qvartam är och så a: p: fermo/ men i
% det du sätter den högre Foten fort/ wänder du honom i lufften /
% så at när du hafwer satt Foten neder / det hälen då står fram;
% Din högre sijda blifwer wäl indragen och borttaggen / böjer
% fördenskull knäät . Denna {\it qvarta} kan du och bruka när
% Fienden wille på dig passera / at du den samme med din udd
% bemöter / så att du bringar ditt korss / dijt der du tillförende
% i string med din swaga war/ så lär hans Klinga gå förbij / och du
% med en hast om han intet passerar gå tillbaka igen / och begifwa dig
% åter i din gvard
The second art is to thrust in fourth is also a: p: f:, but in that
you put your Foot down, you turn it in the air, so that when you have
put your Foot down, the heel is forwards. Your right side is well
pulled in and taken away, bend for this your knee. This fourth you can
use when the Enemy is trying to pss on you, that you in the same with
your point meet, so that you bring your cross, to where you were in
the bind with your weak, his blade should pass, and you with alacrity
if he does not pass go back again, and move back into guard
