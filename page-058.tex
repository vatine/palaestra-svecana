\newpage

\scan{058}

\newpage

% Translation below

% högre Axel / at han griper med sin styrka effter din swaga / och
% geer sig i det samma utan til öfwer sin högre Arm blotter / så
% caverar du i samma [???] och söker Ansichtet komma innan til din
% Klinga / på det sättet kan man altijd stöta qvarta utan till /
% allenast at kroppen har hatur tertia / stöt altså per qvartam eller
% tertium öfwer hans h¿ogre Arm in. Ocj så när han effter din inwertes
% fint itan jern parerar; Men ginge han med en strak impetu utur
% presenza behöfwes intet at du föllier med korset till hans swaga/
% utan stöt långt med jämbd Klinga öfwer fort.

\chap{The Thrust in Fouth}

% Det äro fyra manier till at stöta qvarta. Den första är  a" p" f"
% med halfwa kroppen / och den högre Foten jempt för sig fra uht / i
% denna qvarta / skall du intet som somblige giöra / hålla Hufwudet
% tillbaka / utan häldre upphöja fästet något till din diffesa /
% särdeles emot secund; Men i det / at fästet med högre hande
% kommer så högt op / måtte man med Ansichtet nederböjt åht
% Armen / att du på det sättet seer din Fiende bak om ditt fäste
% allenast med höger ögat; fördenskul blifwer och Armen innan til
% något angulerat det med du pa hans werkning kan taga bättre achta /
% och således parera hans Klinga ifrån dig/ hwilket så wäl a: p: f:
% som i den {\it girerte} och oasierande i qwarta wil nödvändigtvis
% wara observerat; Den ófre delen af din kropp skal du tillijka dit
% Ansichte jämpt fram uträckia / ty om du wille hålla Hufwudet
% tillbaka / skulle det ntet wara så säkert / erföre / ju wijdare den
% samma ifrån styrkan af Klinga / ju mehr och mehr blifwer hon öppen
% och blotter / såsom och kan du en håller på sådant sät din stöt
% förlängia / at du til din Fiendes kropp hinner; Här hoss wijdare
% skal du i acht taga/ det du stöten i {\it qvarta} inter förwder /
% förr n du har fåt {\it mensuram} / och först . udden / sedan
% Hufwudet / handen tillijka korset gieck til has swaga/ så öpnar du
% dig intet så bittijd i den högre sijdan / en heller låter hans udd
% ofwan til {\it angulum} mindre lufft/ hwilket emot de undergvard
% särdeles will tagas i avht / och om du kommer med din stöt tillijka
% med Klingan i {\it qvarta} diupare in . och effter {\it proportion}
% måtte du din kropp mehr och mer halbera; En i det at stöten har
% råkat / så af Klingan och kropper tillijka fulkomlig har wändt sig/
% hwilket med en hast uti ett {\it tempo} i det Foten sättes in. Emot
% de medel och undergvard blifwer {\it qvarta} öfwer hans Klinga med
% försänkt korss / och upphögd udd / til hans högre Axel stötter
% allena at kroppen blifwer wänder.

% Den andre arten at stöta per qvartam är och så a: p: fermo/ men i
% det du sätter den högre Foten fort/ wänder du honom i lufften /
% så at när du hafwer satt Foten neder / det hälen då st/r fram;
% Din högre sijd ablifwer wäl indragen och borttaggen / böjer
% fördenskull knäät . Denna {\it qvarta} kan du och bruka när
% Fienden wille på dig passera / at du den samme med din udd
% bemöter / sa att du bringar ditt korss / dijt der du tillförende
% i string med din swaga war/ så lär hans Klinga gå förbij / och du
% med en hast om han intet passerar gå tillbaka igen / och begifwa dig
% åter i din gvard