\newpage

\scan{021}

\newpage

% Translation below
 % derföre intet kan komma med din klinga i preens; Derföre giör du en
 % mezza cavation, af orsaak / att den samma hastigare blifwer
 % effectuerat, och in??? [ inskter ??? ] med det samma kroppen utur
 % presens hans udd. Denna mezza cavtion blifwer intet altijd uti
 % förste tempo utan mestendehls i den andre eller tridie motion, hwar
 % effter mens' och stringerande blifwer när eller fierran tagen/
 % practicerat och anbracht.

% Gynnes fördenskull / som ofwan är omtalt/ att man med alla
% cavationer ( de som sker a tempo, om man fulfuliförer sin effect )
% råka kan / eller man kommer sa långt fram/ att man i den
% nestfölliande rörelsen träffa kan; hwilket är dett rätta manier att
% operera;

% Men blifwa cavationer giorde utan tempo, måtte man offta/ emedan
% mens' är så när/ med en retirade cavera, der men man i medlertijd
% caverar, af Fienden intet blifwer råkat.

% Derföre så offta din Fiende är in larga, eller i nermare mens' med
% en inträdning/ och föllier din swaga/ måste du cavera, ty uti det
% swaga är ingen force eller krafft att stringera eller
% parera. Caverar din contrapart utan retirering, när du redan har
% wunnit med din styrkia hans swaga/ giörs intet behf att du caverar/
% utan går i en jembd linea ända fram/ så att du far derigenom ett
% tillfälle på en annan sijda att lædera/ emädan han går med sin swaga
% i din styrkia/ och blifwer hans Tempo mucket större som igenomgår /
% än den som går ända fram; En den som caverar måtte giöra twå Tempo
% ett i cavaden, dett andr i stöten; Der emot den som stöter giör en
% motion i dett han går ända fram / derföre den som will wäl cavera/
% måtte intet gå alt för högt med sin udd/ utan något försänkt / så
% blifwer din contrapart intet så snart din cavation warse/ och när
% cavation är ute/ måste stöten wara ändat. cavera skeer på twänne
% sätt; först de som hålla uht/ caverar man med en Circel/ så får man
% sin Fiendes swaga des bättre/ och är wäl för säkrat för sin Fiendes
% uhthållande och stöt; Men der din Fiende är i motu, caverar man med
% en oval/ och lärer din Fiende beswärlogen kunna parera någon stöt/
% han må wara så hastig som han will/ dock röres intet mera än främsta
% leden der man hafwer Werjan uti.

\chap{What Schemes are, and how they are used}

% Chiamater eller de bedräglige ypningar och blott/ är i dett
% 5. Cap. under contratempo påmint/ huruledes man skall gå emot dem/
% föllier altså hár huru man och säkert A Tempo kan buka dem.

% Chiamater äro de blott / sedan man genom fördel af contrappostur /
% eller med en linea recta har fåd den Enga mensuram / att man d/a går
% med sin udd / tager sin Fiendes Klinga / och gifwer på sig en falsk
% ypning/ så att man låckar sin Fiende till att stöta.