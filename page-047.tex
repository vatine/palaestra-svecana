\newpage

\scan{047}

\newpage

% Translation below

% dett din Fiende intet binder dig utan till / fördenskull håller du
% ända fram för dig udden och högre Axeln lijft i en jämbd linea,
% bättre lijtet för högt / än att man skulle försänka henne; En der
% har hon lijka egenskap som prima, att hon der som swagast är; Och
% kunde din fiende lätteligen få dig till att parera, och derpå
% passera unde din Klingar bort: Der han skulle utan till Der han
% skulle utan till Artaqvera, måste du cavera i secunda, dock utan
% Approchering eller annalkande/  dett wari sig / att du i dett samma
% han nalkas / att du kunde stöta de inwärtes stöter / kan du med
% qvarta lätteligen defendera dig . af orsak att Klingan blifwe mycket
% längre uthsträckt. Secuna är och fördelachtigt nog / sig der uti att
% gvardera, när underdelen af kroppen blifwer wäl borttagen / at
% Fienden intet kan räcka honom . med mindre han icke tillförende
% stänger hans Klinga / hwilket ändå blifwer swårt att giöra /
% emedan man med ringa motion och möda der uti kan Cavera; Men dett är
% och swårt att länge oppehålla sig der uti.

% Öfvertertia wäl sträcketer och med hela kroppen giord / är den
% bästa gvardia sin Fiendes Klinga der uti söka och stringera / emädan
% man snart der utur kan gå i qvarta eller secunda, effter som tijd
% och lägenhet sig gifwer.

% När då i denna tert' din Klinga på något sätt blefwa occuperat,
% måtte du med udden i Undertertia per lineam obliqvam gå till Jorden
% / och din kropp och knää som tillförede wor öfwerbögde / måtte
% aldeles retireras så åft du sådan fara frij / och hafwer betagit
% honom den mensur' till att stöta; Och om han då skulle på nytt söka
% att string' din Klinga eller inträda/ kan du honom i det samma
% allena med kropsens öfwerböjning / förutan att röra foten stöta
% och han intet så lätt få din Klinga / emädan udden står åth
% marken / och om han intet weet at bruka den vantagio med sin krop /
% i dett samma han föllier henne effter / blifwer han säkert stötter
% Af orsak de distancier uti förberörde gvardia äro så bedräglige/
% att en menar dett han är långt ifrå sin Fiende / då den andre med
% blotta kropsens öfwerböjning/ mehr än med halfwa Klingan / utan att
% röra foten sig så mycket utur men' tillbakas begifwa kan / och så
% framt den andre icke har gvard' natur förstådt / lär han komma
% honom längre in än han tänkt hafwer: Em så mycket som han gifwer
% sig tillbaka / kan han och gifwa sig fram igen / och med offesa och
% diffesa tillijka gå in på honom / så att hans Klinga intet blifwa
% stringerat / utan icke Fienden kommer honom der med i den enga
% mensura, om han hölle då sine fötter nähr tillhopa och bögde
% kroppen så mycket öfwer som möijeligit wore / så lähr han hinna
% hans udd.