\newpage

\scan{097}

\newpage

% Translation below

% Tert' med uddens försänkning utan till under hans Klinga till hans underlijff. Så blifwer honom hans igenomgång förhindrat Fig. 12. Wille han gå med sin styrkia i din swaga / och utan till stöta med dig tillijka / så låt honom intet widröra din Klinga / utan Cavera och stöt med secunda under hans Klinga Fig. 21 Du kant och träda effter med högra foten och string honom utantill / nät du Caverar. stöt Qvarta a. pie fermo i stringeringen blif med raak Arm för honom / och rör allenast den fremste leden med din Klinga atraverso till hans. Du kant och wäl i string', något stijga till hans kropp / när du wilt der op voltera. På föregående string med porterer Klinga är myckit rådeligt / emot the Chiamater i dett du med en hast / när han wille stöta / förändrar effecten.

\exercise{}

% Ligger di Fiende neder och öppnar sig innan till / så gack honom, per tert' gögt an med något inwersz udd; Rycker han då med klingan eller elliest giör någon rörelse med foten eller Kroppen / såsom han wille innan till stöta / och som då medelst i mens; Så stöt i dett samma Contra tempo Qvart' ofwan till fort a pie fermp, och gack med en hast tillbaka igen / låtandes din udd siunka i Qvarta för effter stöten.

\exercise{}

% Ligger han i seconda; Så string honom utan till; Rycker han derstädes och förer foten i mens' såsom han wille innan till stöta / så stöt qvarta med dett falska steget Contratempo till hans inwersz kropp. Fig. 11.
