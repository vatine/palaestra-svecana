\newpage

\scan{031}

\newpage

% Translation below

% fötterna och kroppen rätt och med gått förstånd weet att regera /
% der med att klingan hennes ämbete och wärken dess bättre kan
% fórrätta / hwilket och fodras / att man i ophäfning den ene foten /
% kroppen hwijlar s/a länge på den andre utan någon Slancering, och
% att den ene foten är altijd befrijat af kroppen / och blifwer
% obeswärat och oförhindrat / så blifwer man sin kropp och Klinka
% mycket mechtigare.

% Hwar hoos är och att merkia / att uti passering i fortsättning
% wenstra foten / att man intet tillijka kommer fram med wänstra
% Axelen utan den högre / hwilket den wänstra accompagnerar, måtte
% fördenskull föhra tån ut pa wänstre foten wäl / så att man bättre
% kroppen fintos / och blifwer man med den samma intet så långt
% tillbaka ; kan man altså bruka Klingans styrkia fast bättre och med
% stöten ( emedan med kroppen på sådant sätt mer till pendicolar ) har
% så långt som han har kunnat giort med högre foten.

% Enteligen att man och i paserande som tillförende korteligen är
% omrört / brede wijd sin Fiendens Klinga | han ligge med henne stilla
% eller intet Continuera och hene altijd föllia skall; Så håller man
% henne i med säkerheet effterfölliande proportion, att man när han i
% gvardian är öfer eller jämpte hans swaga / hwar effter han ligger
% högt  eller lågt / gå in med en stadig Arm / at man honom med
% Klingan når / och som man kommer diuper och diupare in / ju mer 
% strecker man sig der meed; Men ändå att man intet mycket berör henne
% / till des man kommer med korset / der man tillförende war med udden
% / och till hans kropp Continuera och stöta / i dett man lijkwäl hans
% Klinga / der med man sig intet alt för mycket blottar / hwarken
% under eller öfwer sig  / eller till sijdan / intet mer än den
% fortgång som wärkar utaf sig sielf ; Effter samma proportion och
% mått skall man och rätta udden till hans blott.

% Men går han i passaden med sin Klinga öfwer eller under ifrån din
% Klinga / m/atte man i dett samma ögnebleck föllia hans Klinga / att
% lijka som bägge Klingorna woro tillhopabundne / så att den ene
% drager den andra med sig / med en raak Arm / och med fremste ledens
% rörelse med den högre handen.

% Men wore din Fiendes mouvement små och geswinna Cavationer, eller
% muterade / blifwer man med Klingan der med man sig icke sielfwer
% disordonerar jämpt f"or honom / och Avancerar med udden till hans
% kors till des man får mens' till stöten.

% In Summa, att stöta  pie fermo är en art att fechta / och passera en
% annan / och den som har bägge wettenskaperne kan taga hwilkerdera
% han bäst befinner sig vara skickader och benägen till; En till
% passering hör en stor skickeligheet / stadigheet / geswindhet /
% gottt jugement och resolution, whlket intet finnes hos whar och en, 