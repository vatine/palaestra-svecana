\newpage

\scan{011}

\newpage

% Translation below

\chap{On the Foundations of the Art of Fencing, it's Beginning and
actual Science}

Fencing is composed of two things, offence and defence, which both are
taken to fullness and executed from 4 Guards or main
positions\sidenote{hufvudstycken}. They are Prima, Secunda, Tertia and
Quarta.

Prima is the position or stance when one has unsheathed one's Rapier
and turn the edge or sharp upwards that was downwards, palm of the
hand outwards, aiming the point towards one's counterpart.

Secunda is when you allow the sharp to fall back enough that your
rapier is flat and your off-hand high.

Tertia is called when your hand without any forcing will be held
naturally without turning to neither siden.

Quarta is when your hand or arm has been bent enough that your palm is
pointing straight up.

\chap{How a blade should be partitioned}

The blade is in four equal parts divided. namely, the part closest to
the guard is the Strong of the blade and no thrust or cut can be so
forceful that it cannot be parried or deflected. Thus this is called
the Main or the String of the blade.

The next part of the blade is a good deal weaker, but it can
nonetheless still protect against your enemy's weakest protect and
defend yourself. Thus it is called the Half-strong. The first half of
the blade is for defence and can for this serve.

The third part serves for offence, since it is too weak for
defence. it is nonetheless divided into two halves, such that one half
and the first part of the weak is only serviceable for
cuts. meanwhile, the other and outmost is for thrusting. although they
can both be used for cutting, no cut, beat or maiming can be done than
with the part that is between teh two weaks.
