\newpage

\scan{049}

\newpage

% Translation below

% Cap 20.
% Hwad som i gemen / så wäl långe angulerte som retirerte eller
% stackote guardien både i offesa och deffesa förmå och innehålla.

% Dett kan en som har wettenskap utaf fächtning / med sin Klinga
% arbeta som han wil / men intet deste mindre af den Cognition / af
% styrkan och swagan Mensura och tempo/ en god effect der utaf giöra/
% likwäl är en guard' bättre/ och öppnar mindre än den andre / och kan
% man med den ene säkrare / än med den andre gå i mensur' / särdeles
% när Klingan blifwer fördt som sig bör.

% Fördenskull sträckia somblige Armen så mycket som dem möjeligit är
% / och hålla Klingan jämpt för sig / så att ifrån Axeln till udden
% gier en jämd linea. Andra hålla Klingan angulerat, Armen försänkt
% något för sig åth knäät / med handen i tertia / eller utan till åth
% secunda wändt. En dehl liggia och med retirerar Arm och jämbd Klinga
% / alltså att han en jämbd linea af udden till Armbogan Formerar.

% Hwad den fste wedkommer med utrechter Arm, / och em jämbd Klinga /
% således att hon synnes gifwa jämbd linie: är ett sådant manier,
% emädan hon håller sin Fiende långt ifrå sig / warsamt nog / utan
% dett att hon är swår och Klingan mer än i den andre gvardia utaf den
% distance som emellan handen och kroppen blifwer förswagat / och
% warder mycket förr utaf Fienden funnen; Hwar till behöfz en god
% opsight / der med man behåller henne frij.

% Hwilken som weet dett samma att practicea, är dett sannerligen
% Fienden ett stort Empeschement, derföre att udden är honom så nähr
% / dett han intet kan gå så diupt in / att han till stöten hinner
% Mensuram / kan en håller (så framt han intet tillförende har
% Klingan och rycker henne utur Presenza) wäl utan fara gå fram för
% sig.

% Om han än skulle med sin styrka / söka att instöta i din swaga /
% lär han ändå der med intet stort uhträtta / emädan att blotterna
% äro små och ringa; I lijka måtto om han stötte inder / blefwo han
% och lätteligen ofwan