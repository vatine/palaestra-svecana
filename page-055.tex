\newpage

\scan{055}

\newpage

% Translation below
% eller under-tertia muterar, eller ligger der med stilla och stöter
% in, när man har mensuram och tempo utan till hans Klinga. Muterar
% han i en heel Circel ifrån sin högra ofwan effter till sin wänstre
% om / så tags mens' och stöt ( i det samma som han far neder till
% sin högra sin sinkos ) secunda med ligation emot hans Klinga innan
% till hans kropp.
or  under-third mutates, or lies still and thrusts in, when you have
measure and tempo outside his Blade. If he mutates in a whle Circle
from his right above after to his left, take measure and thrust (in
the moment taht he goes downadrs to his right his lowered) second with
a bind against his Blade inside to his body.
%                          Ligger han i secunda / eller muterar der uti / och
% hans Klinga seer något till hans wenstra, så gack honom med secunda,
% i en jembd linea till hans kors / när du då åst i mensura / så
% gack i en hast öfwer hans Klinga / och stöt secunda med ligation i
% ett tempo utan till under hans Klinga; förutan ligation blifwa
% Secunda,
Lies he in second, or mutates to it, and his BLade seems to be
somewhat to his left, go at him with second, in an even line to his
cross, when you then are i nmeasure, go quickly over his BLade, and
thrust second with a bind in one temp outside under his BLade, without
a bind Second becomes
%                Tertia Qvarta utan till under Klingan brukade / när du
% innan till emot en hög eller låg gvardia / med eller utan jern
% Finterar eller derstädes stöter / han då med sin Klinga fahr öfwer
% till sin wenstra utur presena / och der uti pareringen, hwarken
% innan eller utan till öfwer Klingan gaf någon blott / eller genom
% hans inträdning
Third Fourth outside under the Blade used, when you on the inside
against a high or low guard, with or without steel Feint or then
thrust, he then with his Blade go over to his left out of presence,
and then in the parry, neither inside or outside over the BLade gave
an opening, or by his stepping in,
%                          / string' / Fintering stöta / med styrkan innan
% till i din swaga indela / eller du har gått utan till emot honom /
% och han med cavation din Klinga will parera, din hand i det du under
% hans kors går igenom / i samma situ, hwar uti hon tillförende war,
% så att udden giör allena effecten, stöter tillijka med kropsens
% försänkning secunda a pie fermo;
bind, Feint thrust, with the strong on the inside in your weak, or you
have gone on the outside against him, and he with a disengage your
Blade wants to parry, your hand in thet you under his cross goes
through, in this situation, wherein it was, so that the point alone
makes the effect, futher thust with the lowering of the body second a
pie fermo.
%                                                      I det samma du går tilbaka igen
% söker at string' för effterstöten / eller gack med passaden i en
% jembd linea wäl berått under hans Arm till hans högre sijda med
% secunda, ditt Ansichte måste komma innan till din Klinga. Utan til
% när Fienden i en jembd linea med sin styrka genom din halfwa eller
% hela swaga / utan eller innan till / förutan eller med cavation
% tillijka med passaden,
Immediately you go back again seeking to bind for the after-thrust, or
went with the passage\sidenote{passaden literaly means 'the passage',
nor sure what's meant here} in an even line well intended under his
Arm to his right side with second, your Face must be on the inside of
your Blade. On the outside when the Enemy in an even lne with his
strong in your half or whole strong, on the out or inside, without or
with a disengage equally with the passage,
%                                     geck till ditt bröst eller Ansichte / brukar
% du secunda längst med den högre skänkel tillijka med honom uti ett
% tempo under till hans högra / i det du hans Klinga utan någon
% parering ofwan till bort passerat, och din hand i högden som hon
% tillförende war / der med du af den linie af din Arm / hwilken du i
% wendningen jempt under hans Klinga bringar / blifwer des bättre
% förswarat.
went to your chest or Face, you use second along your right
ankle\sidenote{check actual translation of ``skänkel''...} equally
with him in one tempo below his right, in taht you his Blade without
any parry above has passed away, and your hand in the same height as
it was, thus you with the line of your Arm, which you in the turn
evenly under his Blade bring, you are better protected\sidenote{or defended}.
%                Du förfaller och i ett arbete med kroppen diupt / och
% förwarar dig för den impetu hans Klinga / men du måste intet göra
% det alt förbittijda / der med han intet låter siunka udden / En då
% står Hufwudet i största fara; Men när det skeer i juste mens'
% blifwer du wäl försäkrat.
You fall in one work withthe body deep, and defend yourself from the
impetus of his Blade, but you must not do it too early, so he won't
lower his point. If so the Head is in the greatest danger. But when it
is done in proper measure you are well insured.
%                                        Så framt han i stället för stöten allena
% med sin wenstra skänkel Finterade, i secunda eller i en annan hög
% gvardia innan till caverade, och du wore så långt i mens' att du med
% kropsens försänking kunde undfly hans udd . kan du med blotta
% wendningen i secunda så w'al med passaden som a pie fermo fort
% stöta.
If he insted of the thrust alone with his left ankle Feinted, in
second or in another high guard disengage to the inside, and you were
so far in measure that you with he body's lowering could escape his
point, you can with the turn alone in second well with the passage as
with a pie fermo thrust.

% Är der hos at merkia / när han har dig emot en undergvardia
% string', och går i det samma af din Klinga / willindes giöra ofwan
% till en Fint, att du intet förfallee när udden har nådt sin högd /
% utan i det samma då när hon begynner at willia få den / ty har du
% der försummat / giör du bättre at du i det hon ehrnådt sin högd
% med korset / ( så framt du så diupt i mens' åst) wäl betäckt / i
% qvarta fort passerar och uti ett arbete continuerar, udden under
% hans högra Arm / eller elliest der han gifwer sig blott. Men på
% Squinci, eller de stöter som komma ofwan effter fallandes /
% m'atte du intet förfalla, utan tillförende lijtet retirera dig /
% så kunna de i pareringen bättre blifwa judicerade, och stöterne
% säkrare anbrachte.
If it is to note, when he has you against a low guard bound, and goes
in the same off your Blade, wanting to make a feint above, that you do
not fall off when the point has reached its height, but in the instant
that it wants to achieve it, because if you have ignored, yo uwould do
beter when it has reached its height with the cross, (as long as you
are in so deep measure) well covered, in fourth quickly
pass and in one movement continue, the point under his right arm, or
otherwise where he gives an opening. But on {\it Squinci} or those
thrusts that come from above after falling, you should not fall away
but just slightly retreat, so should they be in the parry better be
judged and the thrusts more safely brought on.

% De innan till öfwer blotten anbelangande / blifwer secunda med eller
% utan Cavation stötter: Utan Cavation skeer det igenom en inwersz
% tempo,

As concerns the openings on the inside above, they are with second
with or without Disengage thrusted. Without Disengage it is done with
inverted\sidenote{This could be ``inwards''.} tempo,
