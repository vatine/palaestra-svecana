\newpage

\scan{071}

\newpage

% Translation below
% \chap{Föllier altså Tertia med des Bruk / Egenskaper och Positeur,
% såsom ofwanbstående Figur N 3. utwisar.}
\chap{Follows thus Third with its Use, Characteristics, And Posture, as above Figure No. 3 shows.}

\exercise{}
% Ligger han långt och hafwer udden innan till / så string hans Klinga
% utan till per tertiam (Fig. 3.) ligger han stilla så stöt med en
% hast utan till per tertiam med uddens försänkning till hans högre
% Axel A. p. fermo (4)
If he is extended and has the point on the inside, bind his BLade on
the ouytside in thrird (Fig 3.) if he remains still then thrust with
haste on the outside in third, with a lowering of the point to his
right Shoulder A. p. fermo (4).

\exercise{}
% String' din Adversarium innan till / caverar han så giff grant acht
% uppå tempo i dett han giör sin cavation att du stöter tert. utan till
% öfwer hans Klinga till hans bröst / udden blifwer något försänkter
% (Fig. 4.)
Bind your Adversary on the inside / if he disengages then pay large
attention to tempo in that he does his disengage that you thrust in
third on the outside over his Blade to his chest, the point is lowered
slightly (Fig 4.)

% Caverar han i dett du stöter Tert. öfwer hans Arm / så förwäxla med
% en hast utur Tert. i Qvart. och stöt A. p. fermo så wähl utan som
% innan till hans kropp.
If he disengages when yyu thrust in Third over his arm, change quickly
out of Third into Fourth and thrust A. p. fermo either on the out- or
inside to his body.

\exercise{}
% \chaptitle{Huru finter här giorde i Tert. med påfölkliande stöter.}
\chaptitle{How feints are done in Third with follow-on thrusts}
% Ligger din Wederpart i Tert. med något ophögd och utwersz wender
% udd, / så bindt honom innan till (Fig. 1) cavera hastigt / och giör
% honom finten utan till; men med en battute af högre foten / dok
% intet diupt / gack i dett samma åter igenom / och giör finten innan
% till / parerar han till sin wenstra / så Cavera och stöt per tertian
% A. p. fermo öfwer hans högre Arm. (fig 4.)
If your Opponent is in Third with smewhat elevated and outwards-turned
point, then bind him on the inside (Fig 1.) disengage rapidly, and go
him the feit on the outside, but with a stomp of the right foot,
though not deep, go in the same through, and do the feint on the
inside, if he parries to his left, then Disengage and thrust in third
A. p. fermo over his right Arm (fig 4.)

\exercise{}
% Ligger han långt / så string'. honom innan till / Moverar han sigh
% intet / så giör honom der medh Styrkian emoot hans Swaga en Fint
If he is in a long position, then bind him on the inside, if he doesa
not Move, do him then a Feint with the Strong against hos Weak.
