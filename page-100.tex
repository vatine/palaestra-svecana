\newpage

\scan{100}

\newpage

% Translation below

% till med halfwa qvarta och kroppen halberat; I dett han åter will gå tillbaka / så stöt honom utan till per tert', inann till per qvartam; Men bägge parering förrätta med stadig Arm; Allenast med fremste ledens rörelse af din högra hand; Item, stöter han utan till så parera honom hans stöt med din styrkia af Klingan / och träd i dett samma utan till med din högre fot atraverso wäl på honom in / så att din öfwerdehl af kroppen blifwer skränkt och fast halberat; Armen blifwer förkortat / din Klinga behålt i din linie och passerat geschwindt fort / och stöt secunda öfwer hans Klinga till hans öfwerdehl af kroppen Fig. 20. Du kant och så emot dhe resolute stöter bruka volter och förfalla

with half fourth and the body halved. In that he seeks to retreat, thrust outside in third, inside in fourth. But both parries should be done with steady Arm. Only with the front-most joint of your right hand. If he thrusts outside then parry his thrust with the strong of your blade, and simultaneously step outside with your right foot sideways\sidenote{Original has ``atraverso''} well in on him, so that your upper body is bent and well halved, the Arm is shortened, your Blade kept in your line and passed very fast, and thrust second over his Blade to his upper body. Fig. 20. You can also against the robust thrusts use circling and retreating.

\exercise{}

% See till att du kommer honom småningom oförmärkt i mens', och gack med din Klinga i en jämbn linie: Så snart du hafwer nådt mens', du ställ dig i ett stadigt postur; Ligger han ändå stilla / så gack något långsamt igenom / och tag med din styrkia hans swaga / ( Då lärer han utan twifwel i ditt tempo stöta) så snart han stöter öfwer din Arm något lågt / så cavera i dett samma / och stöt qvarta med dett falska steget under hans Klinga / till hans högre bröst. Fig 13.

Ensure that you eventually get into measure without being noticed, and go with your Blade in an even line. As soon as you are in measure, stand in a stable stance. If he remains still, then go somewhat slowly through, and take with your strong his weak (then he will without doubt thrust in your tempo) the moment he thrusts over your Arm somewhat low, disengage instantly and thrust fourth with the false step under his Blade to hist right breast. Fig. 13.

\exercise{}

% Ligger han med en flygande Klinga så att hans hand / der uti han förer werjan / är altijd i motu, förandes udden högt fig. 18, Och söker så att kasta qvarta öfwer din högra Arm utan till in; Om du skulle så simplement der parera, blefwo du utan all twifwel råkat. Alstå giör du bättre att du gifwer dig utan till öfwer din högre Arm öppen / stat i dett samma med fötterna ferm, och giff gran acht uppå hans rörelse / i dett samma han kastar in qvarta öfwer din Arm / att du då straxt förfaller med din kropp så diupt som möjeligt är / och wänder din högre hand i secund' wähl opp; Dock utan någon utsträkning / så lärer du råka honom på halsen öfwer hans Klinga / din wenstre hand förer du wäl fram till din högra / så att du förhindrar honom hans igenomgång. Fig. 20.

If he has a flying Blade\sidenote{High?} so that his hand, when he is moving his rapier, is always in motion, keeping the point high fig. 18 And seeks to throw fourth over your high Arm outside in. If you were to simply parry, you were without doubt injured. Thus you would do better that you give yourself outside over your right Arm open, stand with your feet firm, and pay due attention on his movement, in the instant he throws fourth over your Arm, that you then retreat with your body as deep as possible, and turn your right hand well up into second. Although without any extension, you will touch him in the throat over his Blade, your left hand you bring well before your right, so that you deny him his going through. Fig. 20.

\exercise{}

% Du kant och / i dett samma / när han inkastar förwända din hand i
% qvarta. och voltera henne tillijka öfwer hans högre Arm. Fig. 19.

You can also, in the moment, when he throws in turn your hand in
fourth and circle the blade equally over his right Arm. Fig. 19.

\exercise{}

% Men der han med sin styrka skulle instöta hos din swaga / måste du i
% dett samma Cavera, och stöta qvarta innan till a. p. fermo eller med
% dett falska steget. Fig. 11.

But were he to thrust with his strong to your weak, you must instantly
Disengage and thrust fourth inside a pie fermo or with the false
step. Fig. 11.

\exercise{}

% På ett annat sätt. Ligger han med hög Klinga / och rörer altijd sin
% hela Arm / sökandes der igenom att komma småningom i mens' Fig. 18
% Giör honom i dett en half stöt till hans bröst; Men sätt inte foten
% långt uht / om du alsintent

In another way. If he lies with a high Blade, and always moves his
entire Arm, seeking thus to eventually get into measure Fig. 18 Do him
a half thrust to his chest. But do not extend your foot far, if you
aren't
