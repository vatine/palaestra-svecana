\newpage

\scan{105}

\newpage

% Translation below

% bak med korset för mera säkerhet skull något högre / parerar han den
% förrige finten så att han blifwer öfwer sitt kors blott / så stöt
% med angulerat secunda om ophögder udd innan till hans
% öfwerlijff. Fig. 10. Altså och när du stöter Qvarta innan till / att
% han den samma / som sagt är wille parera. Parerar han intet efter
% finten, så stöt Qvarta ända fram. Wore du i finten med ditt kots
% något högre från hans Klinga afgått / så låt ditt kors / när han
% intet parerar, i stöten åter så mycket undersiunka.

back with the cross somewhat higher for safety / if he parries the
previous feint so that he becomes open over his cross, then thrust
with an angled second with raised point inside to his upper
body. Fig. 10. Also if you thrust Fourth inside, that he the same, as
was said wants to parry. If he doesn't parry after the feint, then
thrust Fourth all the way. If y0ou were in the feint gone somewhat
higher with your cross, then let your cross sink, when he doesn't
parry, in the thrust much sink under.

\exercise{}

% Giör honom finten utan till med Contrapostur till sin högra Axel |
% parerar han till sin högra / så cavera och stöt innan till qvarta
% a. pie fermo eller angulerat qvart' öfwer hans Klinga a. pie fermo;
% Så kan du i dett samma med en omwridning twinga hans werja utur
% handen; Med ditt kors gack bak högt / far han med sin Klinga bak
% öfwer sig / så stöt Secund' till hans inwersz kropp och bocka dig
% wäl. Fig. 21. Både desze föregående finter / blifwa och så på en
% cavation giorde. Item med en contra-Cavation förrättade.

Give him the feint outside with counterposition to his right Shoulder,
if he parries to his right, then disengage and thrustr inside to the
fourth a pie fermo, of an angled fourth over his Blade a pie
fermo. You can in that instant with a twist force his blade out of the
hand. With your cross go back high, if he goes with his blade back
over him, then thrust Second to his inside body and bend
well. Fig. 21. Both these preceding feints are done on a
disengage. With a counter-disengage executed.

% \exercise{Contratempo huru de blifwa tagne.}

\exercise{Counter-tempo how they are taken}
% General regel öfwer Contratempo. Wille han innan till din swaga med
% ett insteg string, eller i dett samma stöta / så cavera och stöt
% utan till tert' Eller / går han med Klingan temmeligen högt / så
% passera under hans Klinga bort / och stöt secunda till hans högra
% sijda under hans Klinga. Fig. 5. Men på kastande stöter måste man
% intet förfalla / utan sig tillförende något lijte retirera, så
% blifwa dhe i pareringen så mycket bättre judicerade.

General rule of counter-tempo. If he seeks to bind your weak on the
inside with a an instep, or in the same thrust, then disengage and
thrust outside in third. Or, if he goes with the Blade relatively high,
then pass under his Blade away, and thrust second to his right side
under his BLade. Fig. 5. But on throwing thrusts you must not
[förfalla], but just slightly retreat, then they are in the parru so
much better judged.
