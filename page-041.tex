\newpage

\scan{041}

\newpage

% Translation below
% stort der med uthrättas / särdeles emot dem som i denna konsten äro
% wäl förfarne / hwarken på en eller annan ohrt att giöra / så framt
% en sådan battut skeer i en lång distance, har du ingen orsak att
% dubitera, i betrachtande att din Wederpart intet kan hinna dig: Och
% om han en sådan battute giorde i mensura, kan du honom ( i dett ditt
% tempo är geswindare ) i samma omento stöta / eller wijsa att du will
% stötta i samma blottt / och sedan gå till de andre ypningar som han
% geer sig i diffesa, (emädan han intet mehr än en ohrt / utan ypning
% den andre) defendera kan / och lähr altså öfwer sin battute , war
% med han mente att reta dig / och gifwa sig tempo, blifwa
% oförmodeligen skelfwer bedragen / och uti fara satter [sätter???]

large is through this done, especially against those who are n this
art well experienced, neither to be done in one place or another, as
long as such a bash is done in a long distance, you have no reason to
doubt, since your Opponent cannot catch upi. And if he did this in
measure, you can him (in that your tempo is quicker) in the same
moment thrust, or show that you will thrust into the same opening, and
then go to other openings tah he gives himself in defence, (since he
cannot\sidenote{At a guess, the original text has a missing ``cover''
here} more than one place, without opening thee other) can defend, and
will thus over his bash, with wich he intended to tease you, and give
himself tempo, becomes without doubt self betrayed, and sets himself
in danger.

% Emädan du som låg stilla / mycet bättre än han som war i motu, hans
% rörelse judicera kan Hwarutaf kommer att Finterne lyckas mycket
% bättre när Fienden är i motu, än elliest när han ligger stilla.

However you who did no t move, much better than he who was in
movement, his movement can judge. From which the Feints will succeed
much better when the Enemy is moving than when he is at rest.

% Somblige giöra Finterna med kroppen och med Klingan; Men dhe bringa
% henne intet mycket fram för sig; Orsaken derföre / att de i
% pareringen intet blifwa råkade / och sedan när Fienden har giort
% Caduren i dett samma stöta kunna.

Some do Feints with the body and with the Blade. But they bring it not
far in front of them. The reason being that they in the parry will not
be reached, and then when the Enemy has done the Cadur, in that moment
thrust.

% Detta manier går an / när man hafwer att giöra med en som är rädder
% / eller oförfaren i denne konsten / elliest lär man intet stort
% uthrätta emor dem samma som här uthinnan äro wualförfarne En när
% Klingan intet kommer fram för sig / så är dett wist / att han intet
% råka kan; Måtte du fördenskull intet movera dig / dett ware sig / att
% du i samma tempo, då han Finterade, stöta will; Eller och i dett
% samma Contrafintera, låsz som du ilt stöta / så lär han sig till
% diffesa pr{\ae}cipipera. Är han i de tankar att du wilt taga
% Contratepmo, så bekommer man derigenom ett önskeligit tillfälle att
% l{\ae}dera. Dett kallas att stöta med contrafint; En den som först
% Finterar blifwer bedragen / och kunna de contrafinten emot alla
% Finter, der ingen fullkommelig tempo och mensura blifwa giorde och
% stötte.

This manner works / when you are dealing with someone who is afraid,
or unversed in this art, otherwise you will not complete anything
great against those who are in this well-versed. When the blade is is
not in front, it is sure, that he cannot reach. You should not move
for this, it being the case, that you in the same tempo, that he
Feinted, should thrust. Or in the same Counterfeint, pretend that you
will thrust, to incline him to defence. If he is in throughts that you
will take Contratempo, you create through this a wished-for moment to
injure. This is called thrusting with counterfeint. In this the first
who Feints is betrayed / and could you counterfeint against all
Feints, no whole tempo or measure is done and thrusted.

% Somblige äro och i Finterne / som gå med Klingan för sig / och när
% Fienden will parera / draga de åter henne tillbaka / och bringa
% sedan stöten med en Slancering. Denne ahrt att Fintera / är ibland
% alle andre den odugeliste; I dett / att när Klingan intet mer än en
% motum giöra skulle / blifwer här igenom tree giorde / som alle äro
% hwar annan emot.

Some are in Feints, who go with the Blade ahead of them, and when the
Enemy parries, draw it back again, and then bring the thrust with a
counter-thrust. This art of Feinting, are among all of them the most
useless. In that, when the Blade should do no more than one movement,
here three are done, who all undo the previous.

% Den förste / i hwilken Klingan blifwer bracht fram för sig; Den
% andre / när Klingan blifwer dragen tillbakas igen. Den tridie och
% störste / när hon äter med stöten blifwer in-Slancerat / och tager
% Fienden intet i achta att hans

The first, i which the Blade is brought forward. The second, when the
Blade is drawn back again. The third and largest, when it is wit the
thrust is counter-thrusted\sidenote{I am starting to doubt that
'Slancering' is actually counter-thrust}, and if teh Enemy does not
pay attention that his