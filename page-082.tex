\newpage

\scan{082}

\newpage

% Translation below
% den förrige linien för sig neder / och kortare i mens'; Giör så
% honom Finten innan till per Qvartam och giff wäl acht huru han
% parerar; Parerar an intet / så stöt per qvartam derstädes med
% passaden af wenstra Skenkelen. Fig. 3.

\exercise{}
% Påtycker din Fiende utan till / emot din swaga / och du har
% försummat Cavation, så tryck honom hårt emot ( i dett / lått hans
% Klinga fara / gack honom med Finten i Contrapostur; will han intet
% parerar så stöt der på i en motyion per avartam till hans inwersz
% kropp a.p. fermo eller med passaden af wänstra skenkelen / parerar
% han så tag tempo i acht. Fig. 9.

% \Exercise{En Lection emot Stringering.}

% När du åft innan Stringerat så lått din udd i en hast nedersiunka /
% gack med din Klinga straxt opp igen / och see till att du får med
% din styrka hans swaga; String honom hans klinga utan till / så snart
% han wille Cavera under din klinga / så gif gran acht uppå Tempo i
% dett samma han går igenom / att du träder med din wenstre fot in på
% honom / och stöt innan till med qvart' åth hans bröst. Fig. 9.

% \exercise{Chiamater med eller utan Retirader giorde.}

% Stringera honom utan till / drag i dett samma din klinga till dig /
% så att hans kommer under din klinga / och din öfwerdel af kroppen
% lått något siunka tillbaka; Syöter han innan till / så bruka dett
% fakska steget. Observatio. De? Chiamater som skee med Retirering
% blifwa brukade när Fienden är i full mensura den andra utan
% retirering, när han intet rätt mens' har till stöten.

% Chiamater på ett annat sätt; Åft du honom utan till hans klinga|
% eller han hade dig derstädes string, så retirera dig med klingan och
% kroppen; öppna dig utan till öfwer din högra Arm / fast i den halfwa
% Secund', och fär fästet wäl utan till / der med att du dig icke
% annorstädes öppnar: Om han wore dig med sin klinga tämmeligen når /
% så kan du wäl sättia din wenstra fot något tillbaka; då kommer du få
% mycket mer utur mens', stöter han ythan till per Tertiam äfwer din
% klinga in / så cavera och parera honom hans stöt med wensatre handen
% / nedan till din högre sijda sinkos / och stöt med Secund, contra
% tempo i en hembd linea till hans öfwerlijff. Fig 13.

\exercise{}

% Annorledes: Gifwer han sif utan till någorlunda blott / och ligger
% med sin Klinga bak något neder / och med udden öfwer sig / så att an
% gofwer en triangel då hålt din udd ungefehr i Qvart' under hans
% Styrkia af hans Klinga / så snart han med sing Klinga på din swafa /
% eller der på wille instöta / så bruka dett falska steget / och stöt
% qvart tillijka med honom utan till öfwer hans högre Arm. Du kant
% och/ när du äst i mens' med qvart utal till stöta / och med wenstra
% handen taga hans Werja utur handen. Fig. 15.

% \exercise{Denna Fig. 10. beteknar huru en Angulerat secunda blifwer anrättad}

% Ligger din Fiende med lång Klinga för dig / och med udden utwers
% wender / så string' honom innan till / ligger han ändå stilla / och
% intet will mvera sig / så giör honom en sint innan till per qvartam;
% Parerar han till sin wenstra något under sig / så att han innan till
% öfwer sitt Kors blifwer öppen / så cedera honom derstädes med din
% KLinga i dett han begynner att parerar, och stöt så med en Angulerat
% Secunda innan till hans öfwerlijf; Men med Kroppen
