\newpage

\scan{082}

\newpage

% Translation below
% den förrige linien för sig neder / och kortare i mens'; Giör så
% honom Finten innan till per Qvartam och giff wäl acht huru han
% parerar; Parerar an intet / så stöt per qvartam derstädes med
% passaden af wenstra Skenkelen. Fig. 3.
the previous line down, and shorter in measure. Do him then the Feint insie in Fourth and pay well attention how he parries. If he parries not, then thrust in fourth with the passing of the left Ankle. Fig 3.

\exercise{}
% Påtycker din Fiende utan till / emot din swaga / och du har
% försummat Cavation, så tryck honom hårt emot / i dett / lått hans
% Klinga fara / gack honom med Finten i Contrapostur; will han intet
% parerar så stöt der på i en motion per qvartam till hans inwersz
% kropp a.p. fermo eller med passaden af wänstra skenkelen / parerar
% han så tag tempo i acht. Fig. 9.
If your Enemy presses on the woutside, against your weak, and you have ignored Disengage, then press hard back, in this, let his Blade more, get at him with the Feint in Contraposition. If he does not parry then thrust in a single motion in fourth to his inside body a pie fermo or with the passing of the left ankle. If he parries then take tempo into consideration. Fig 9.

% \exercise{En Lection emot Stringering.}
\exercise{A lession against binding}
% När du åft innan Stringerat så lått din udd i en hast nedersiunka /
% gack med din Klinga straxt opp igen / och see till att du får med
% din styrka hans swaga; String honom hans klinga utan till / så snart
% han wille Cavera under din klinga / så gif gran acht uppå Tempo i
% dett samma han går igenom / att du träder med din wenstre fot in på
% honom / och stöt innan till med qvart' åth hans bröst. Fig. 9.
When you have bound on the inside then let your point drop with speed,
go with your Blade slightly up again, and ensure that you get with
your strong his weak. Bind him his blade on the outside, as soon as he
wants to Disengage under your blade, then pay much attention to Tempo
in the instant ge goes through, that you step with your left foot in
on him, and thrust in the fourth to hist chest. Fig. 9.

% \exercise{Chiamater med eller utan Retirader giorde.}
\exercise{Invites with or without Retreats done}

% Stringera honom utan till / drag i dett samma din klinga till dig /
% så att hans kommer under din klinga / och din öfwerdel af kroppen
% lått något siunka tillbaka; Stöter han innan till / så bruka dett
% fakska steget. Observatio. De? Chiamater som skee med Retirering
% blifwa brukade när Fienden är i full mensura den andra utan
% retirering, när han intet rätt mens' har till stöten.
Bind him on the outside, pull in the same instant your blade towards
you, så that he comes under your blade, and your upper body let sink
back slightly. If he thrusts on the inside, then use teh false
step. Note. Those Invites that are done with retreat are used when The
Enemy is in full measure the other without retreat, when he does not
have the right measure for the thrust.

\exercise{}
% Chiamater på ett annat sätt; Åft du honom utan till hans klinga|
% eller han hade dig derstädes string, så retirera dig med klingan och
% kroppen; öppna dig utan till öfwer din högra Arm / fast i den halfwa
% Secund', och för fästet wäl utan till / der med att du dig icke
% annorstädes öppnar: Om han wore dig med sin klinga tämmeligen när /
% så kan du wäl sättia din wenstra fot något tillbaka; då kommer du få
% mycket mer utur mens', stöter han uthan till per Tertiam öfwer din
% klinga in / så cavera och parera honom hans stöt med wensatre handen
% / nedan till din högre sijda sinkos / och stöt med Secund, contra
% tempo i en jembd linea till hans öfwerlijff. Fig 13.

Invites in anoter way. If you have him outside his blade, or he
elsewise have you there bound, then retreat with your blade and body,
open yourself outside over your right Arm, but in the half Second and
move the hilt well outside, so that you do not open elsewhere. If he
were with his blade very close, you can well place your left foot
slightly back, then you will be more out of measure, if he thrusst on
the outside in Third in over your blade, then disengage and parry his
thrust with the left hand, down to your right side bent, and thrust in
Second, countertempo in an even line to his lower chest. Fig. 13.

\exercise{}

% Annorledes: Gifwer han sif utan till någorlunda blott / och ligger
% med sin Klinga bak något neder / och med udden öfwer sig / så att han
% gifwer en triangel då hålt din udd ungefehr i Qvart' under hans
% Styrkia af hans Klinga / så snart han med sin Klinga på din swaga /
% eller der på wille instöta / så bruka dett falska steget / och stöt
% qvart tillijka med honom utan till öfwer hans högre Arm. Du kant
% och/ när du äst i mens' med qvart utan till stöta / och med wenstra
% handen taga hans Werja utur handen. Fig. 15.

In another way; If ge gives an opening on the outside, and lies with
his Blade back and somewhat low, and with the point above him, so that
he gives a triangle the hold your point roughly in Fourth, under the
Strong of his Blade, as soon as he with his Blade on your weak, or
then wants to thrust, use teh false step, and thrust fourth at him on
the outside over his right Arm. You can also, when you are in measure
with fourth outside thrust, and with the left hand take his Rapier
from the hand. Fig. 15.

% \exercise{Denna Fig. 10. beteknar huru en Angulerat secunda blifwer anrättad}
\exercise{This Fig. 10. depicts how an Angled second is done}

% Ligger din Fiende med lång Klinga för dig / och med udden utwers
% wender / så string' honom innan till / ligger han ändå stilla / och
% intet will movera sig / så giör honom en fint innan till per qvartam;
% Parerar han till sin wenstra något under sig / så att han innan till
% öfwer sitt Kors blifwer öppen / så cedera honom derstädes med din
% Klinga i dett han begynner att parerar, och stöt så med en Angulerat
% Secunda innan till hans öfwerlijf; Men med Kroppen

If your Enemy is with a long Blade blade before you, and with the
point outwards turned, then bind him on the inside, if he then lies
still, and will not move, then do him a feint on the inside in
fourth. If he parries somewhat to his left somewhat low, so that he
becomes open on the inside over your Hilt, then cede him this with
your Blade in that he starts to parry, and thrust then with an Angled
Second inside to his upper body. But with the Body
