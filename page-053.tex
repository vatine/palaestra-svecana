\newpage

\scan{053}

\newpage

% Translation below
% at du går med din Klinga till hans blot / träder du med din styrk
% till hans swaga / och Continuerar på sådant sätt der hos / på dett
% att han intet med en angulo kan råka; Derföre du med din klinga
% somblige gånger i stóten går jämpt fór sig / ibland öfwer och under:
% udden bör wändas / hawr effter Fiendes blott eller swaga går
% högt eller lågt. Och merck att emot de nedrige guard' eller rórelser
% / effter giord ligation med hela secunda, uti en angulo med halfwa
% secunda, försänka korset / och udden wäl opp till hans Klinga / emot
% högre gvard' och rórelsen till större defension, hela secunda
% emädan derigenom Fiemdens klinga blifwer bättre bort parerat, warder
% altså stöten utan till hans Klinga giord; Dett wari då, att du
% tillförende med din Klinga har wistas i qvarta, du brukar då föt
% mindre rörelse skull tertia. Uti andre fall blifwer secunda innan
% til löfwer eller under klingan med avancerande högre Axeln på bägge
% s'att a pie fermo, eller och med passaden stötter. Hon kan och
% blifwa anbrachter utan till öfwer Klingan / när han på den
% utwärte Finten intet aldeles parerar öfwer sig / att du i dett samma
% förwänder utur tertia i Secunda, på hwilket sätt / du lär
% /ater komma med dit kors till hans swaga / och med försänkter udd
% stöta utan till öfwer hans Arm in. Så kan du och din Fiende (när du
% har emot en under gvardia stringerat / och han will utan till med en
% lång stöt öfwer din Klinga) stöta i secunda contratempo. Utan
% till under Klingan blifwer secunda med eller utan ligade
% stötter. Med ligation läderar man Fienden på många sätt / med
% större säkerhet som till Exempel, när du din Fiende innan till med
% contrapostur, eller med Finten per qvartam är mäst kommen i den Enga
% mens' / han då utan fotens rörelse caverat, eller utan till en
% angulerat chiamata giorde / och du öfwer haus motion stötte in
% öfwer hans Klinga eller Arm / och han giorde i dett samma en volt,
% williandes stöta under din Klinga in / eller söka att råka
% dig tillijka / eller hålla udden för dig / så lät med en hast din
% udd i hehl secunda siunka / och stöt i samme ligation med en
% wridning / förr än han har ändrat cavation uti ett tempo utan till
% under hans Klinga / så lär du honom i hans förtagnde cavation
% opprimera.

% Man kan och i det samma giöra en contracavation / och de offta / aff
% och till gående med klingan / blifwer honom förhindrat.  Men wore
% du tillförende innan till något diupt kommen / så skränk kroppen i
% ligade under stöter / så mycket / att du med ditt kors kan blifwa
% med hans swaga ; Udden rättar du till hans blott / din wenstra Axel
% såsom och wenstre handen förer du wäl fram / så komer den högre
% sijdan bättre sinkos / och du färdigare att taga honom med dit
% wänsta hand wijd hans fäste. Men skedde denna cavation eller
% chiamata in larga mens' så tag tillförende med ligation bättre
% Mensuram / och stöt med dett samma utan till under hans Klinga /
% eller gack "ofwer i en jämbda linie / och föhr klingan på fintwiisz
% / eller sänck till Mensuram och stött der på smo förr sagt är /
% med ligaden fort. Och ,å du wäl här emot desze Chiamater med en fint
% per Tertiam i en jämbd linie med lijtet jern brede wijd hans korsz
% giöra / så kan du igenom en lijten [ ??? ] en förr komma til at
% lädera / och har du intet behof att fruchta dig / dett han (emdan
% d'a finterat med din swaga i hans swaga) utan till skulle st öta; En
% (för än han med sin udd kommer i Presensa och har sin stöt
% förrättadt) hafwer du redan / dig Cavation och stötet tillijka
% libererat / eller och taga Contra tempo der emot.
