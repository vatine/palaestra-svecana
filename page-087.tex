\newpage

\scan{087}

\newpage

% Translation below

\exercise{}

% Uppå ett annat sätt. Ligger han i den medel-tertia, jempt för sig
% med något ophögder udd / så halt din udd med uhtsträckter Arm
% ungefähr ett qvarter under styrkan aff hans Klinga / och blif med
% din Klinga [bok?] till något högt; Stöter han utan till / så Cavera
% och stöt med ett falskt steg till hans inwertz kropp. Alltså och om
% du hade honom innan till string', och han stötte utan till / med
% korszet så mycket försänkt, att du intet kommer till cavera. så gak
% i dett samma med din klinga öfwer sig / och stöt utan till öfwer
% hans högre arm.

In another way. If he is in the middle third, evenly in front with sligfhtly elevated point, then hold your point with extended Arm roughly a quarter under the strong of his Blade / and bring your Blade ]???] to slightly high. If he thrusts outside, then Disengage and thrust with a false step to his inside body. Thus as if you had bound him inside, and he thrust to the outside, with the guard lowered so much that you do not come to disengage, this go in the same with your blade over it, and thrust outside over his right arm.

% \chaptitle{Föllier altså huru wijda den wenstra handen uti hwarjehanda tillfall är nödvändig och tillåtelig}
\chaptitle{Follows thus if the left hand in every opportunity is needed and allowed}

% Jag föreställer mig uti detta / huruledes kan hända / en eller annor
% lärer finnas / som wijd fechtande med en ensam werja / hålla aldeles
% orådsamt att bruka den wenstra handes; Sombl. utaf en
% sielfbehageligit efftertänkade / hos sig sielwe / såsom och gifwa
% androm sine ståhl / att dett icke woro rådeligt och oprichtigt föra
% sin Action emot sin Wederpart med wenstre handen jämpte klingan;
I am imagining this, if it can happen, that one or another will exist,
who when fencing with a single rapier, holds that using the left hand
is inadvisable. Some from a self-pleasing thought, in themselves, to
give other their steel, that it would not be advisable and honest to
bring their acion against their Oppponent with the left hand along the
blade,
% Hwilke omdömen och tankar jag icke uti alt bifaller / såsom man
% effter tijdens lägenheet, ingalunda den wenstre handen får uthsluta
% eller låta aflägse. Huru kunde iag detta med ståhl tillstå? när man
% är stadd i krig eller andre farlige och swåre händelser / att man då
% skulle läggia den wenstre handen på ryggen / eller och låta henne
% hängia såsom förlammat och död i neder wihd sijdan / hwilket först
% förtager kroppen sin fördelachtighet / och gifwer honom i stället en
% mechta snöppeligit och afstympat anseende:
These opinions and thoughts I do not fully agree with, such that one
must with time and situation the left hand leave out or remove. How
would I this with steel accomplish? When you are in war or other
dangerous and challenging situations, that one would place the left
hand on one's back, or let it hang lame and dead down by one's side,
which first robs the body of its advantages, and instead leaves it
with a mightily disadvantaged and amputated look.
%                                            En skall kroppen hafwa
% något behageligt och mächteliget anhållande / wijd en behörig
% Fecht-postur måtte den ene leden icke wara den andre emot /
% undandragen och förfallen / utan mycket mehra till beredskap och
% behjälpsam. Derföre är dett ofehlbart / att in och efterstöter med
% rätta application af wenster Arm och handen / gifwer mycket mehra
% säkerhet / än man skulle föra werjan allena.
You should hold your body comfortably and powerfully, with a valid
fencing posture one joint must not oppose another, withdrawn and
destitute, but much more ready and helpful. Thus it is unfailing, that
in- and after-thrusts with the right application of the left Arm and
hand, give much more security, than wielding the rapier alone.
%                                              Emedan man i brist och
% återlembnande af wenstre Armen intet kan betäckia och bringa kroppen
% så när tillhopa att man uti stöten icke gifwer fast mera öpning
% derigenom. Alltså effter det man icke med wenstre Arm och handen
% kommo den högre till hielp och undsättning / blefwo mången sielf i
% däran träffat och läderat, der han wijd bijstånd aff den wenstra
% hade elliest timat råka och fullbracht stöten Fördenskull bör
% hos en förståndig Fechtare noga i acht tagas / hwilkendera i stöten
% blifwer blott eller betächt såsom och i hwarjehanda tillfällen
% retteligen wetta bruka den wenstra Arm och handen.
Since one in lack and return of the left Arm cannot cover and bring the body so close that one in the thrust does not give a larger opening through it. Such that after one have not with the left Arm and hand come the right to help and rescue, many have been hit and injured, where he with help from the left would have reached and completed the thrust. For this reason a clever Fencer closely attend, which is in the thrust is opened or covered in all manner of occasions correctly know to use the left Arm and hand.
%                                                    Min mening är
% ingalunda att man entelige skall taga effter eller omfatta sin
% Fiendes klinga / der man tillskyndar sig undertijde snarare afskurne
% och förderfwade händer / än någo fördehlachtigheet. Derföre moste
% man föra wenstre handen bak och icke fram för udden; Och skulle man
% ibland detta utwällia dett lindrigaste / woro ett slag uppå en
% läderhanska fast drädeligare än en stöt i lihfwet;
My point is not that one shall solely reach for or grab one's Opponent's blade, whereby one mostly hasten cut-off and ruined hands, than anything advantageous. Thus one must keep the left hand behind, and not in front of, the point. And should one among this choose the mildest, a cut to a leather glove be mucg preferred to a thust in the abdomen.
%                                                    Hwarföre tyckes
% mig sådant som af en eller annan härutinnan blifwer således
% förebracht icke wara rättmätigt utan ogagneligit / dett man nyttiar
% sig til i bästa uti sådan nödfall / den wenstre handen der lijkwäl
% den högre Gud hafwer tildelt och gifwit så wäl den wenstre som högre
% handen till kropsens hielpsamhet och defension. Hwarföre tyckes
% detta förhinderligit / ens frije händer ihopknyta och sammanbinda
% och så medelst i otrångt måle lembna ens lijff och lefwerne uti fara
% och öppenheet /
Thus it strikes me that one or another that here has been mentioned is
not right but unfavourable, something to be used best in emergencies,
the left hand as well as the right have been by God assigned and given
the left as well as the right hand to the body's help and
defence. Thus it seems that it would be to the detriment, to tie one's
free hands together and thus in tight situations leave one's life in
danger and openness,
%                 då lijkwäl min emotståndare hafwer äfwen så wäl twå
% händer som iag. Är altså detta ett lindrigt inkast / som kunder
% någon här wihd låta förmärkia oc seja att om werjan wore wäl hwäst
% låter man fuller hålla handen tilbakas. Men ingalunda om det så
% händer sig i all fall / att wenstra handen skulle brinkgas der
% igenom i fara der doch mestendels ingen nöd befinner sig woro dock
% lijderligare i handen än i kroppen.
as my opponent also have two hands like me. This is thus a mild
obserbation, that someone can note and say taht if the rapier were
well-sharpened you hold the hand fully back. But nonetheless if it
happens, that the left hand is brought into danger where mostly no
ganger is present, it is better to be injured in the hand than in the
body.
