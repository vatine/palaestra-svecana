\newpage

\scan{018}

\newpage

% Translation below
% här jämpte måtte du hafwa flitig acht på mensura, i betänkande / som
% offta uti ett är gott / är uti en annan mensura skadlig; En om han
% skulle gifwa en ypning med sin kropp / eller med sin klinga i larga
% mensura, är ändok ingen säkerheet ellet wissheet / så länge han
% blifwer med sine fötter ferm. och stilla stående / är ingen tijd att
% råka eller träffa / af orsak / att den rörelsen med kroppen och
% klingan ät mycket hastigare / än med föttren; En för att du har
% bracht din stööt fram med foten / har han fådt tijd at parera, eller
% den samma / mädan han står ferm med sine fötter; kan han bryta
% mensuram och således störa dig / för än du kom i ordning igen; Giör
% du fördenskull bätter och radligare / när han på detta sätt (som nu
% omtahlt är) rörer sig / du i samma tempo med en fot utur den wijda i
% den enga mens' med eller utan Fint inrycker.

You must always mind what measure you are in, as a thing that is good
in one measure may be harmful in another. If he gives an opening with
his body or blade in wide measure, this is not guaranteed or safe as
long as his feet are firmly planted and he's standing still you may
not have time to thrust, as the movement of the body and blade is
faster than moving forward on your feet. And if you have brought your
thrust forwards with foot movement, he may have had time to parry, or
similarly by standing still he may have broken measure and and this
disturb you before you can come back into order. Therefore, better 
and more prudent that you should when he moves in this way (as now
discussed), you in the same tempo move a foot out of wide, into
narrow measure with or without a feint.

% Men när du Mensuram Largam tagit hafwer / och din Adversarius will
% med sin klinga Accomodera sig / sökandes en rx8"orelse / med foten
% och kroppen tillika / eller med foten allena / oansedt / om han med
% foot eller kropp i dett mouvement eller Approchering utur den wijda
% i den enga mens' är occuperat, intet kan parera, hwarföre är sådana
% rörelser (emädan de skee med ingen retirade) rätta tempo att råka
% sin Fiende uti / särdeles om han har gifwit tempo utaf
% oförsichtigheet / såsom han uti ett tempo tillijka icke Avancera och
% eloingera kan..

But when you are in wide measure and your Adversary accomodates you
wiht his blade, seeking a movement, with foot and body alike, or with
the foot alone, either way if he with foot or body in the movement or
approach from wide into narrow measure is occupiet, cannot parry. Thus
these movements are (as they happen without retiring) the right tempo
to rwch your opponent in, especially if he has given tempo by
carelessness, like he within a tempo cannot both advance and elongate.

% Nu på dett att sådant kan hafwa sin bättre fortgång / är utaf nöden/
% att du befinner dig i Contrapostur; En om han skulle ha warit den
% första / som rörde sig/ han då utan twå tempo icke kunnat parera och
% stota/ och du (för än han har parerat) har din stöt redan råkat och
% med samma brutit mensuram, hwilket din Fiende inte hade kunnat giort
% / emädan han rörde sina fötter.

To ensure that this can have better sucess, it is vital, that you are
in Contrapostur. Since if he was the first to move, he will not be
able to parry and thurst in les than two tempos, and you (before he
has parried) have already landed your thrust and in that broken
measure, which your Enemy will not have done, since he moved his feet.

% Elliest kan du och ibland i denna mensura (om han intet skulle röra
% sin foot) stöta; Af orsak / om han gifwer ett tempo utaf
% oförsichtigheet/ wederfars honom dett samma som han han minst tänkte
% uppå/ i dett han intet merkte / att han gaf sin wederpart occasion
% till att stöta/ derföre har han intet kunna a tempo parera, eller
% bryta mens'.

Otherwise you may sometimes in this measure (if he has not moved hs
foot) thrust; The reason, if he gives a tempo by carelessness, it
causes him the same as he had the least considered, in that he did not
notice that he gave his counterpart occasion to thrust, thus he will
not have been able to parry in time, nor break measure.

% Fördenskull måtte man awara försichtig och sluger; En somblige gee
% ibland ett tempo med list/ hwarigenom de reta sin Fiende att stöta /
% och i det samma han stöter / hafwa de uti ett tempo tillijka parerat
% och stött/ hwilka kallas stöta med Contratempo, som offta skeer att
% de bägge råka tillijka / hwilket af den samma som intet har tagit
% Contratempo, eller när han giff tempo, har han warit s/a når i
% mensura, eller och har han giordt sin rörelse så stoor.

Because of this, one must be careful and smart. Some will occasionally
give tempo with cleverness, thus tempting their Enemy to thrust, and
in the moment he thrusts, they have in the same tempo both parried and
thrust, which is called thrusting with Contratempo, which often is
done that they both reach, through wich the one who has not taken
Contratempo, or when he gives tempo, has been close in measure or has
made his movement so large.  

% En att undly denna faran med Contratempo, måste man först hafwa gran
% acht uppå/ om den rörelsen (för än han blifwer giord) är så stor

To escape the danger of Contratempo, you must first have given good
attention to if the movement (when it is being made) is so large
