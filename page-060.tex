\newpage

\scan{060}

\newpage

% Translation below
% Men ditt kors i det samma abaissera, tillijka
% förskränka hans Klinga med wenstre handen / så blifwer han
% nederkämpter / och du äft för honom säker. Dock i alt detta / måtte
% du försänkia kroppen mycket wäl / der med at Armen des mindre har at
% röra sig och giöra någon cadut;
But lower your cross, and also angle his blade with your left hand, he
will be fought down, and youare from him safe. But in all this, you
must lower your body well, 
%                                 På lijka sätt warder passerat utan
% till i tertia, korset blifwer och abaisserat, i betrachtande at man
% i passering altijd när sin Fiendes Klinga blifwa skall / effter
% denna Qvarta hwars natur är fram för de andre de starkeste och bäste
% / sin Fiende der med assailera och twingia ;
In a similar way is passed on the outside in third, the cross is
lowered, taking nto consideration that in passing one shall always be
close to one's Enemy's blade, after this Fourth whose nature is before
the others the strongest and best, one's Enemy with it assault and force.
%                                              Du kan och på ett
% utwärtes hugg med mera styrkia än a pie fermo ingå / och blifwer
% volten med denna qvarta bortparerat, oansedt Fienden med styrkan aff
% Klingan wore med dig alpare, ty i den samme åft du måst unierat, och
% kant utan all möda utur den ene effecten i den andre ko~ma / och om
% det woro nödigt att giöra en añan pardite eller förändrig;
YOu can also on an outside cut with more strong than a pie fermo go
in, and if the circle with this fourth parried away, no matter the
Enemy were with you with the strong of his blade more nominal, since
you must in that moment you must often be unified\sidenote{It is not
clear, at all, at this point, what this actually means}, and can
without all effort from one effect into the second come, and if it
were necessary to do another parry or change;
%                                                            Men du
% måste med din styrka taga hans swaga wäl i acht; En här hafwer
% qvarta med de falska stegen ingen krafft / utan faller med tilfälle
% under hans Klinga sinkos med kroppen. Sluteligen merck i gemeen hos
% stöten i qvarta,
But you must with your strng take hiw weak well into consideration;
Since here, the fourth withthe false steps has no power, but falls
with opportunity under his Blade bent wit the Body. Finally, notice
in general with  the thrust in fourth,
%                  ty när Fienden i sin gvard' eller i stöten öfwer
% sitt kors gifwer blott / så logera edu din udd derstädes i hela
% qvarta / och gack under hans Klinga / så framt det intet är i ett
% under eller mede-läger stött.
because when the Enemy in his guard or in the thrust above his cross
gives an opening, lodge your point then in the whole fouth, and go
under his Blade, as long as it is not in a low or middle position thrusted.
%                               En i sådant förblifwer du med dit
% kors hart öfwer hans Klinga / och söker allena halfwa qvarta med
% ophögd udd till hans högre Axel. Enteligen / blottar han sig under
% korset / så stöt hela qvarta så wäl emot under som öfwer gvard',
% och stöt med försänkt Klinga och korset öfwer hans / hwilket du på
% sådant sätt / med större wisshet och säkerhet / under hans Arm
% in:
In that case you remain with your cross just over his Blade, and seek
only half fourt with raised point to his right Shoulder. Finally, if
he opens himself under the cross, then thrust full fourth both against
a low and a high guard and thrust with lowered Blade and the cross
over his, which you in such a fashion, with larger certainty and
safety, under his Arm in,
%     Men geck han med Klingan något för högt / at du till samme
% skränkning / intet fulkomligen kunde tillkomma / förblifwer du i
% stöten med ditt kors under hans Klinga wälbetäckt / och logera så
% udden i hela qvarta under hans högre Arm in.
But went he with the Blade slightly too high, that you with the same
crouch could not completeley arrive, remain in the thrust with your
cross well covered, and lodge then the point in full fourth under his
right Arm in.

% Föllier nu huru man sig i qvart' på allahanda sätt och manier emot
% sin Fiende bruka skall / och först genom hwad medel han utal till
% öfwer Klingan blofwer nbracht; Parerar han Finten, som blifwer giord
% med Constrapostur, utan till per tertiam till sin högre sinkos / och
% betäcker sig icke aldeles / utan låter korset något siunka / så att
% han gifwer sig en angulum utan till öfwer Armen / så voltera utan
% till öfwer hans Arm / eller stöt henne a pie fermo / då kant du
% om du will taga ifrå honom hans wärja / i det at du med wnestra
% handen / när du har råkat honom innan till hans swaga / sätter
% den samma inder sig till din kropp / ed korset åter i högden uti en
% empito ifrå dig bortwrijder / så blifwer han twungen at släppa
% henne.

% Och merckt här jämpte denna qvart med det falska steget / är för
% andra god at bruka / när du innan till med din stöt i qvarta
% befruchtar at din Fiende skulle giöra en angulum. Item, när han
% derstädes giorde ett tempo / och geck i det samma med sin udd utur
% präsenza / så at du intet kan stöta fort hoss hans Klinga / utan du
% icke tillijka bled läderat / kan du med denne wredne wvart' med
% kroppen komma utur presenza hans udd / din stöt går ända fram /
% och hans förbij; Men när han gifwer ett temp i sitt läger / måtte
% fotens rörelse wara der hos / der med han i din stöt intet bryter
% mensuram, och du blifwer öfwerstötter / hwilket och kunde skee / när
% du låter hnom tempo at mutera sin effect; En så framt den samma
% som volterar och
