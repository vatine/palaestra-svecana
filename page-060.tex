\newpage

\scan{060}

\newpage

% Translation below Men ditt kors i det samma abaissera, tillijka
% förskränka jans Klinga med wenstre handen / så blifwer han
% nederkämpter / och du äft för honom säker. Dock i alt detta / måtte
% du försänkia kroppen mycket wäl / der med at Armen des mindre har at
% röra sig och giöra någon cadut; På lijka sätt warder passerat utan
% till i tertia, korset blifwer och abaisserat, i betrachtande at man
% i passering altijd når sin Fiendes Klinga blifwa skall / effter
% denna Qvarta hwars natur är fram för de andre de starkeste och bäste
% / sin Fiende der med assailera och twingia ; Du kan och på ett
% utwärtes hugg med mera styrkia än a pie ferm ingå / och blifwer
% volten med denna qvarta bortparerat, oansedt Fienden med styrkan aff
% Klingan wore med dig alpare, ty i den samme åft du måst unierat, och
% kant utan all möda utur den ene effecten i den andre ko~ma / och om
% det woro nödigt att giöra en añan pardite eller förändrig; Men du
% måste med din styrka taga hans swaga wäl i acht; En här hafwer
% qvarta med de falska stegen ingen krafft / utan faller med tilfälle
% under hans Klinga sinkos med kroppen. Sluteligen merck i gemeen hos
% stöten i wvarta, ty när Fienden i sin fvard' eller i stöten öfwer
% sutt kors gifwer blott / så logera du din udd derstädes i hea
% qvarta / och gack under hans Klinga / så framt det intet är i ett
% under eller mede-läger stött. En i sådant förblifwer du med dit
% ors hart öfwer hans KLina / och söker allena halfwa qvarta med
% ophögd udd till hans högre Axel. Enteligen / blottar han sig under
% korset / så stöt hela qvarta så wäl emot under som öfwer gvard',
% och stöt med försänkt Klinga och korset öfwer hans / hwilket du på
% sådant sätt / med större wisshet och säkerhet / under hans Arm
% in: Men geck han med Klingan något för högt / at du till samme
% skränkning / intet fulkomligen kunde tillkomma / förblifwer du i
% stöten med ditt kors under hans Klinga wälbetäckt / och logera så
% udden i hela qvarta under hans högre Arm in.

% 

