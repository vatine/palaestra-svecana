\newpage

\scan{117}

\newpage

% Translation below

% så att hans halfwa styrkia kommer i din halfwa styrkia / så förwänd
% din hand med en hast utur qvart i secunda, och stöt henne Angulerat
% utan till öfwer hans Arm a pie fermo, eller med passaden. Der han
% intet hade tagit effter Finten, utan blef stilla liggiandes / stöter
% man qvarta ända fram öfwer hans Klinga til hans högre Bröst.

So that his half strong comes in your half strong, then turn your hand
with haste out of fourth into second, and thrust Angled outside over
his Arm a pie fermo, or with the passage. If he idd not take after the
Feint, but remained still, thrust fourth all the way over his Blade to
his right Breast.

\exercise{}
% Ligger din Adversarius i ett långt läger / och hafwer wändt sin udd
% innan til så lägg din Klinga under hans i sprospettiva hans kors /
% gack i der samma igenom och giör honom en hast stöt utan till öfwer
% hans högre Arm / med en hast Cavera och giör honom en Fint innan
% til; Parerar han effter din Fint, så gack åter geswindt igenom / och
% stöt qvartam utan til öfwer hans högre Arm til bröstet; Men hade han
% samme Fint med en förhängiande qvarta meder parerat / förwänder du
% utur qvarta i tertia; Korset låter du siunka / ophäfewr udden wäl /
% och stöter til hans högre bröst a. p. fermo eller med passaden.

If your Adversary is in a long position / and have turned his point
inside then place your Blade under his in [sprospettiva???] cross, go
through and do him a quick thrust outside over his right Arm, quickly
disengage and do him a Feint inside. If he parries your Feint, then go
again speedily through, and thrust fourth outside over his right Arm
to the chest. But were he the same Feint parried with a hanging
fourth, you turn out of fourth into third. Let the cross sink, elevate
the point well, and thrust to his right breast a pie fermo or with the
passage.

\exercise{}
% Der din Fiende skulle i din Fint ( som giordes i qvarta ) Cavera med
% det samma / och hölt altså uth / eller stötte något högt / så
% förwänd din hand i secunda / förfall under hans Klinga / och stöt
% med secunda till hans högre bröst a pie fermo eller med passaden;
% Men der han geck ( i det du giorde Finten ) under din Klinga igenom
% / och låt sitt kors något siunka / så förwänder du utur qvarta och
% stöter med secunda ända fram till hans högre Axel a pie fermo.

When your Enemey were in your Feint (which was done in fourth)
Disengage immediately, and thus held out, or thrust somewhat high,
then turn your hand into second, fall under his Blade, and thrust
second to his right breast a pie fermo or with the passage. But were
he went (as your were doing the Feint) through under your Blade, and
let his cross sink slightly, turn out of fourth and thrust second all
the way to his right Shoulder a pie fermo.

% \exercise{Huru man skall bruka Cavationer, Contracavationer, och
% Ricavationer.}
\exercise{How one should use Disengages, Counter-disengages and Ricavazione}
% Ligger din Wederpart med sin Werja i en hög secunda, och gifwer sig
% inwersz öppen / så stringera honom innan till / Caverar han i det
% samma under din Klinga / så cavera tillijka med honom / at du kommer
% med din Klinga åter innan till / och i cavation rycker på honom
% något lijtet in; Går han i det samma åter under din Klinga igenom /
% så fall honom med din styrkia på hans Klinga / och stöt med en hast
% qvart öfwer hans Klinga till hans wenstre bröst / med ett insteg aff
% högre foten.
If your Opponent lies with his Rapier in a high second, and gives
himsefl open on the inside, then bind him inside, if he in that
Disengages under your Blade, then disengage with him, that you again
come with your Blade inside, and in the disengage somewhat jerk in on
him. If he then goes under your Blade through, then fall with your
strong on his BLade, and thrust quickly fourth over his Blade to his
left breast, with an instep of the right foot.

\exercise{}
% Men der han ( i det samma då du geck utan til hans Klinga ) med en
% hast gingo under din Klinga igenom / så stöt i samma tempo secunda
% innan til hans bröst; Med ditt kors bak till gak wäl högt
But when (in the instant you went outside to his Blade) wit haste goes
under your Blade through, then thrust in the same tempo second inside
to his chest. With your cross rear go well high.

\exercise{}
% Der han icke under din Klinga Caverade / när du geck på hans / utan
% farer bak öfwer sig / så cavera i en hast och stöt secunda under
% hans Klinga a: p: f:
When he doesn't Disengage under your Blade, when you went to his, but
go back over himself, then disengage with haste and thrust second
under his blade a pie fermo.

\exercise{}
% Lägg dig långt i tertia med udden något inwertes; Wille han då med
% ett insteg i mensuram utal til string' din Klinga / så cavera för än
% han rörer din Klinga / och stöt utan til öfwer hans Arm per qvartam
% till hans högre Axel / med ett insteg af wenstre foten a: p: f
Lie long in third with the point slightly inside. If he then with a
step into measure seek to bind your Blade outside, then disengage
before he touches your Blade, and thrust outside over his Arm in
fourtj to his right Shoulderwith an instep of the left foot a pie
fermo.

\exercise{}
% Lägg dig åter som tillförende / string' han dig utan avancement /
% och kommer dig intet fullkommelig i mensuram / så cavera och giör
% honom utan til en fint med qvarta hos hans klinga; parerar han til
% sin högra så passera och stöt med angulerat secunda öfwer hans
% klinga: Du kant och stöta secunda under hans klinga a: p: f:

Lie again as before, if he binds you without advancement, and does not
come fully into measure, then disengage and do him outside a feint in
fourth to his blade. If he parries to his right then pass and thrust
an angled second over his blade. You can also thrust second under his
blade a pie fermo
