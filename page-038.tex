\newpage

\scan{038}

\newpage

% Translation below

% än på sin Klingas styrka Funderar, lär fördenskull intet mycket
% achta Klingan / utan altijd mera betänkt huru han kan behålla sin
% Kling frij / drager altså henne aldeles tillbaka / hwilket
% förorsakar att hans blott blifwer derigebom mycket större / och du
% (efter giord stöt / för än han har bracht sin Klinga fram igen)
% kunnat Salvera dig.

than on his Blade's strength, will because of this pay less attention
to the Blade / but always consider how he can keep his Blade free /
will thus draw it back / which causes his openings through this to
become much larger / and you (after done thrust / before he brings his
Blade forth again) can Save yourself.

\chap{Here we ask / which is better or more advantageous / to stand
with feet and body bent / or straight and upright}

% Hoos denne frågan är till betrachtandes / att kroppen emot Klingan
% är så muycket större / on skönt Klingan är uti långden något / så är
% hon lijkwäl der emot temmeligen smal / hwarföre har Klingan mera
% difficult af de Motioner och blott som sig desz större befinna uti
% diffesa, whilket förer med sig desze skiäl och raisoner: Att man
% skal stå rakk på fötterna: På det hufwudet är då intet i så stor
% fara som elliest: Menniskian är och då färdigare och hurtigare att
% röra sig: Står med mindre möda, än den som hafwer stät sig Contraint
% och krokot.

Considering this question / that the body compared to the Blade is so
much larger / even if the Blade is long / it is comparatively narrow /
thus the Blade has more difficulty in the Motions and openings that
primarily exist in defence, which brings the following reasons; You
should stand straight on the feet; In this, the head is in less danger
than otherwise; The Human is also thus more ready and quicker to move;
Stand with less effort, than the one who is standing contrary and bent.

% Här uppå är att swara: Att somblige ofwantald rationer äro sanne /
% och somblige intet; En först / den som står raak / han är i större
% fahra / och till offesa intet så beqwäm och wäl skickader; En lijka
% såsom den behöfwer i diffesa en stor Motion, altså kan han en helle
% (så framt han icke kröker kroppen och gier honom fram öfwer) icke
% wål förlängia eller uti hastigheet stöta; Så befinner han sig och
% när han står raak / disunierat, förswagat med Klingan / oförmögen
% och krafftlös.

To answer this: That some of the abovementioned reasons are true / and
some not; At first, the one who stands straight he is in more danger,
and to offence less comfortable and well placed; Equally as it needs
in defence a large Motion, he can thus (as long as he does not
bend the body and give himself forwards) not elongate or with speed
thrust; Thus he finds himself when he stnds straight, disunified,
weakened with the Blade, unable and powerless.

% Der emot / när han sine blott wäl förstår / och weet taga dem sinkos
% / och den Angulum som han i öfwerböijningen formerar, är på fötterna
% wäl accomoderat, lärer i dett han förnederar sig / mycket mera wara
% försäkrat / af orsak han mindre gifwer ypning / och med ringare
% Motion af sin Klinga defenderar sig / Forcen är mera unierat,
% hwilken union giör en hurtigheet och geswindheet.

On the other hand, when he well understands his openings / and knows to
take them back / and the Angle that he in the bending forms, is well
on the feet accommodated, will in that he lowers\sidenote{The original
Swedish could be 'lower' or it could be 'disparage', I suspect the
former is correct here} himself, be much safer, because he less gives
openings / and with less Motion by his Blade can defend / the Force is
more unified, which makes for rapidity and speed.

% När man tillförende med sådan böijning af kroppen genom flijtig
% öfning blifwer tillwänder / giör dett i fortgången att man blifwer
% färdigare / beqwämare och säkrare att ingå / och sig utan disordre
% defendera kan. Stöten räcker längre in / i dett att kroppen redan är
% fram öfwerböjder / och fördenskull utan

When eventually with such a bending of the body through diligent
practice becomes accustomed, going forward it means that you become
more ready, comfortable and safer to step in, and without disorder can
defend. The thrust reaches further in, in that the body is already
bent, and even without
