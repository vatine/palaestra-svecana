\newpage

\scan{104}

\newpage

% Translation below

% jämlikt / och båcka dig wäl / så är hans Klinga ligerat; Stöt i dett
% samma om du hafwer mens', med tert' eller halfwa secunda med
% ophögder udd under hans klinga till hans bröst; Men geck han på
% ligation åter öfwer sig / och wille utan till stöta / så gack
% tillijka med honom och stöt secunda ( i dett han går öfwer sig )
% wähl betäckt utan till under hans Klinga; Men wille han hålla emot
% din ligation, så stöt öfwer per qvartam utan till fort
% a. p. fermo. Ligger han på ligation stilla / kan du per tertiam
% öfwer hans högre Arm stöta / och dig under med din wenstra hand
% förwara. Du kan och wäl med ditt kors gå lågt / der med att han
% ingen qvart' kan på dig under din klinga voltera.

evenly, and bend well, his blade will be bound. Thrust the moment you
have measure, with third of half second with elevated point under his
Blade to his chest. But if he in the bind again went above, and seeks
to thrust outside, then go with him and thrust second (when he goes
above) well covered ouside under his blade. But if he were to hold
against your bind, then thrust over in fourth outside rapidly a pie
fermo. If we stays still in the bind, you can in third thrust over his
right Arm, and below with your left hand defend. YOu can also go low
with your cross, so that he is unable to circle onto you under your
blade.

% \exercise{Emot Mutering utur ett underläge}
\exercise{Against Changes from a low position}

% Muterar din Wederpart nederåth / så hålt i begynnelsen din udd till
% hans kors ofwan före / när du har mens', så ligera hans Klinga / och
% stöt utan till i ett arbete med halfwa secund' under hans klinga in
% hos hans swaga till hans högra Axel med passaden, eller stöt jämpt
% öfwer hans swaga / med förhögder udd / och korset försänkt så mycket
% att du kommer der med på hans klinga / kroppen sänkes wähl der med
% att Armen des mindre blifwer rörder till hans högre Axel och är dett
% lijka gott om du håller handen i tert' eller i Qvart', allenast att
% du håller en jämbd linea öfwer hans klinga i Underdelen af kroppen
% drag wäjl tillbaka.

If your opponend changes downwards, then initially hold your point to
his cross and above, when he has measure, then bind his Blade, and
thrust outside in one effort with half second under his blade in by
his weak to his right Shoulder with a passage, or thrust evenly over
his weak, with elevated point, and the cross lowered enough that you
get it on his blade, the body well lowered such that the Arm is moved
less to his right Shoulder and it is equally good if you hand is in
thrird or Fourth, as long as you hold av even line over his blade in
the Lower part of the body pull well back.

% \exercise{Huru man emot dhe medle och nedrige Gvardien fintera och stöta skall.}

\exercise{How to feint and thrust against the middle and lower guards}

% Ligger han i ett medel- eller underläger / så string' honom innan
% till / giör honom der på en Fint derstädes till hans medel-lijff i
% en jämbn linea; parerar han / så stöt per teriam med korset under
% hans swaga / och udden ophögder till hans högre Axel a. pie fermo
% eller med passaden. Och är här ibland de medel- och underläger ingen
% annor åhtskillnadt / än att du emot desze i string', med något
% förhängande af udden / och så mycket Fiendes klinga är lägre string'
% öfwer hans swaga. Men skedde paradem med en empito till Jorden /
% giöres intet behof att du blifwer med ditt kors på hans klinga /
% utan stöt långt per tertiam fort.

If he is in a middle or lower position, bind him on the inside, then
do him a Feint to his midriff in an even line. If he parries, then
thrust third with the cross under his weak, and the point lifted to
his right Shoulder a pie fermo or with the passage. And here is among
the middle and low positions nothing differentiated, other than
against these in a bind, with slightly hanging point, and how much the
Enemy's blade is lowly bound over his weak. But were the parry done
with much beating to the Ground, there is no need to remain with your
cross on his blade, but thrust long and quick in third.
