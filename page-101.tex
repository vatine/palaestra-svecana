\newpage

\scan{101}

\newpage

% Translation below

% rörde foten / woro mycket bättre / allenast med Kropsens öfwerböjning; Skulle han i det samma stöta utan till öfwer din Klinga / så drag med en hast din Kropp tillbaka igen / och cavera i det samma / tag med din Styrka hans Swaga; Förer han sin Hand i stöten högt / så stöt med en hast Flanconnade utan till under hans högre Arm per qvartam a pie fermo Fig. 16.

\exercise{}
 % Du kant och så / der han skulle faram med sin Arm aff och till i Högden Fig. 18. allenast med din Krops öfwerbörjning / stöta honom i Armen eller Handen.

\exercise{}

% Stöter han högt / särdeles till din högre Axel / är intet rådsamt at man volterar; En om man då skulle voltera, blefwe man stötter i Ryggen: Dergöre är det bättre at man står något höget med Kroppen / och gifwer sig öfwer Klingan något öpen / så at han hafwer orsak at sättia högt an; I det samma tempo han stöter / förwänder du din hand i Secunda och förfaller wäl med din Kropp under hans Klinga / och passerar med Secunda under hans klinga bort. Fig 22.

\exercise{}

% Ligger din Advers' i en jämbn tert' fram för dig / så string' honom hans Klinga innan till / kastar han då med en hast sin Klinga öfwer din / i mening att komma med sin Klinga utå till din KLinga / så achta noga när han oplyffter sin Klinga / at du med en hast går ifrån din wenstra til din högra under hans Klinga / så lät du åter komma innan till hans Klinga; Nät du caverar
