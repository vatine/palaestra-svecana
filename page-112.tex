\newpage

\scan{112}

\newpage

% Translation below

% Hwad anbelangar Wärjan eller sielfwe Instrumentet, så frågas här /
% hwilketdera är bättre eller fördelachtigare / at fächta med en lång
% eller stackot klinga; Hwar till är att swara uppå det somblige mena
% / at när man hafwer en lång Wärja / så kan man icke allenast hålla
% sin Fiende längre ifrå sig / uthan och bättre med stöten hinna den
% andre / som intet hafwer den längden; Hwilket och lätteligen kan
% låta sig giöra / besynnerligen aff dem som hafwa en lång taille, äre
% wäl disponerte /
As pertains to the Rapier or the Instrument itself, here we ask, which
is better or more advantageous, to fence with a long or stout
blade. Where we thus answer what some mean, that when you have a long
Rapier, you cannot only keep your Enemy further away, but also with
the thrust outspeed the other, who does not have taht length. Which is
easily done, especially of those who are long-limbed, are well
disposed,
%                  hafwa en god Methode at stöta / ett godt maner uti
% sine Cavader att avancera, sampt weta att behålla sin Klinga frij /
% så at när cavation är uthe / måste Stöten stå; Desze hafwa aff
% Klingans längd en stoor avantagie; Men det emoot / dhe samma som
% intet hafwa desze Wetenskaper / låter ofehlbart / en lång Klinga
% wara den mehra till hinders / en fördeelachtigheet; Ty ju länge en
% Wärja är / iu mehra Swicht och Swagheet gifwer hon / derföre till
% defension är hon intet dugelig / ?? han en stadre fast nyttigare och
% beqwämare för dem som lagt sig på defense, och hafwa ett godt manier
% at passera; Hwarföre är det bättre at man hafwer lagom långe Klingor
have a good Method for thrusting, a good habit in their Disengages to
advance, and also know to keep their Blade free, so that when the
disengage is out, the Thrust must be. These have from the Blade's
length a good advantage. But opposite, them who do not have this
Science, will infallibly, allow a long Blade to be more of a
hindrance, than an advantage. Because the longer a Rapier is, the more
Föex and Weakness it will give, therefore not suitable for defending /
[???]\sidenote{Not actually sure what the manuscript says here} he a
steady more useful and comfortable for those who are on the defense,
and have the good habit of passing. Therefore it is better to have
Blades that are long enough\sidenote{The original Swedish uses
``lagom'', which is a word that means somewhere between ``enough'' and
``sufficient''},
% / som wijd pasz kunna räckia en medelmåttig Man up till Naflan och
% Klingans bredd der emoot bör i proportion aff längden uthi Styrckian
% wara som en god Mans tuma / och så fort som hon går tukk sin swaga /
% måste hon altijd affkorta i Brädden / så at hon gifwer en reen
% baft[??] från heele swagan till halfwa Styrckians begynnelse: En är
% eb Klinga alt för bred / så fångar hon mycket Wäder med sig / hwar
% igenom motion blifwer des långsammare gjord. En Klinga måste och
% wara på bägge kanterne wäl skärp: och hwasz / at hon der bättre
% binder /
that roughly reach a medium-proportioned Man to the belly button and
the width of the Blade should in proportion to the length of the
Strong be the width of a good man's thumb, and as soon as it goes into
the weak, it must always diminish in Width, so that there is a clean
curve from the whole weak to the start of the half Strong, If a blade
is too wide, it will catch the Wind, whereupon all motion will be more
slowly done. A Blade must also have both edges sharpened, to better
bind,
%          håller fastare och icke skrinnar eller slinter ifrån. En
% heller måste en Klinga wara alt för tung; En man måste altijd laga
% at Handen är Herre och rådande öfwer Klingan / och icke Klingan
% öfwer Handen. Fästet måste wara stadigt giordt: Handkafwelen intet
% för mycket tiock. En ju mer Fingren äro från hwar ändre / ju mindre
% krafft hafwa de at grijpa och fast hålla / er emoot ju mera de äro
% tillhopa /
hold more fast, and neither skate nor slip away. Neither must the
Blade be too heavy. One must always ensure that the Hand is Lord and
rules the blade, and not the Blade over the Hand. The hilt must be
firmly made. The grip not too thick. Since the further the Fingers are
from each other, the less strength they have to grip and hold firm,
but likewise the closer they are together,
%            ju större Styrckia hafwer Handen aff dem. Det är fuller
% intet utan / at defension intet består aff Fästet utan i Klingan /
% och särdeles i det hwasze aff Klingan; Men så är hugg och stöt kan
% komma igenom hwar annan / då fästet i sådan fall komma wäl till pasz
% / emädan man icke alle stöter hafwer på ett snöre eller uthi en
% linie.
the more Strength the Hand has from them. It is also fully no, that
defense is not in the Hilt, but in the Blade, and especially in the
sharp of the Blade. But as cuts and thrust can come through each
other, then the hilt in these cases can be well used, since not all
thrusts have on a string or in a line.
