\newpage

\scan{093}

\newpage

% Translation below

% wenstra hand wille grijpa ffter din Klinga då voltera med en hast
% Qvarta och Cavera med din wenstra till din högra under hans wenstra
% hand / då låter han taga förgiäwes; Stöt med dett samma i ett tempo
% qvart utan till öfwer hans wenstre Arm / till hans wenstre bröst; Du
% kant och Qvarta långt a. p. ferno stöta öfwer hans wenstre Arm.

left hand seeks to grab for your Blade, then circle quickly to Fourth and Disengage with your left to your right under his left hand, then his grab will be for nothing. Thrust in the same in one tempo outside over his left Arm, to the left of hist chest. You can also long over host left Arm thrustr a pie fermo.

\exercise{}
% Ligger din Contrapart med sin wenstra hand bakom sitt kors / gak
% honom an med din Klinga utan ill / och giör honom derstädes en fint
% öfwer hans högre Arm till hans wenstra öga / så snart han wille din
% Klinga med wenstre handen ofwan före till din wenstre sijda borttaga
% / så cavera med en hast ifrån din högra till din wenstra öfwer hans
% wenstre hand så lärer han förfara med sin hand / stöt der på med
% tert' utan till hans bröst. Fig. 4.

If your Counterpart has his left hand behind his cross, go at him with your Blade on the outside, and do him then a feint over hist right Arm to his left Eye, the instant he tries to take your Blade with the left hand from above tries to move your blade to your left side, disengage quickly from your right to your left side over his left hand, then he will go astray with his hand, thrust then in third outside to his chest. Fig. 4.

\exercise{}
% På ett annat sätt; Ligger han åter med wenstre handen bak om sitt
% kors / så string honom innan till / och giör honom en fint der
% sammastädes öfwer hans wenstra hand till hans Ansichte / så snart
% han din Klinga med wenstra handen ofwan till sin högra wille
% borttaga / då Cavera med en hast öfwer hans wenstre hand / så griper
% han felt / stöt i dett samma qvart till hans bröst Fig. 7.

In another way. If he is again with his hand behind his cross, then bind him inside, and do him a feint over his left hand to his Face, the moment he wants yo move your Blade with his left hand above to his right, then Disengage quickly over his left hand, then he will grasp wrongly, thrust instantly fourth to his chest Fig. 7.

\exercise{}
% Ligger han i förhängiande Secunda och förer sin wenstra hand fram med
% pannan / wäntandes att du skall stöta in på honom så giör honom en
% fint med Contra postur uti en jembd linea åht hans bröst; så snart
% han wille parera med sin wenstra hand till sin högra sinkos / så
% cavera ifrån din wenstra till din högra omkring hans wenstra Arm /
% och stoöt utan till öfwer hans wenstre Arm Secunda.

If he is in a hanging Second and brings his left hand forth with his forehead, waiting for ou to thrust in at him, do him a feint with Counterposture in a smooth line to his chest. The moment he seeks to parry with his left hand down to his right, then disengage from your left side to your right around his left Arm, and thrust outside over his left arm in Second.

\exercise{}
% Emot dhem som ähro wane att parera med wenstre handen / moste man
% intet giöra resolverte longa utan halfwa stöter / särdeles finter ty
% dhe angulerade stöter skaffa honom mycket att giöra / besynnerligen
% dem som intet fundament hafwa utaf fechtning utan sielfwe utaf
% naturen hafwa wänt sig: Men dhe samme lära en häller stöta / med
% mindre dhe tillförende hafwa med deras wenstre hand fått Fiendens
% Klinga och giöra dhe samma / stora och långsamme motioner; En i dett
% dhe taga med wenstre handen effter Klingan så kommer wenstre Axeln
% fram; Men der emot går den högre Axeln tillbaka igen / på hwilket
% sätt stöterne gå mycket långsamma.

Against those who are used to parry with hte left hand, you must not make commited long but half thrusts, especially feints as the angled thrusts give him much to do, in particular those who are not well-founded in fencing but simply have by nature become accustomed. But they will also not thrust, with less than they have already caught the Enemy's Blade and do the same, large and slow motions, When they graps with the left hand after the Blade the left shoulder goes forward. But likewise the right Shoulder is pulled back again, and in this way the thusts are much slower.
