\newpage

\scan{057}

\newpage

% Translation below
% stötter / altså att den samma / men öfwerdelen af den högre Axelen / den
% fremste eller högre Foten corresponderar, hwilket och i qvarta a: p:
% f:, är att observera / allenast att man med Ansichtet / i dett samma
% (mädan handen blifwer innan till wänder) seer bak, eller utom Klingan
% fram; Udden låter du i stöten någt siuncka / der med de under-blotten
% emot hans förfallning / blifwa bättre förswarade: Ditt högre Knää,
% måtte du wäl i stöten utsträckia och kröka / handel wäl fram för sig
% utfördt / och n''ar stöten är giord / hurtigt och geswindt åter igen
% gå tillbaka. Med lugationskeer, det fördenskull / att man är bättre
% betäckt med hela kroppen till hans högre sijda. Och så när du emot de
% medla och under gvard / med udden upp / och korset sänckt / utan till
% öfwer swagan hans Klinga instöter.

% item. när du emot de höga gvardier / kroppen tillförende skränker /
% och der på utan til öfwer Klingan med passaden ingår / läderar du
% Fienden med den samme uti tertia; F''orst utan till öfwer Klingan /
% n''ar han han det sammastädes i en hög eller lägre gvard' / på din
% string' eller Fint per Tertiam med den Vantagio af Contrapostur
% blifwer giord / ligger stilla och låter dig de blott till st"oten
% stå öpne / eller innan til caverar / der stöter du effter Fiendes
% gvard' och rörelse i en sådan proportion.

% 