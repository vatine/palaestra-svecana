\newpage

\scan{057}

\newpage

% Translation below
% stötter / altså att den samma / men öfwerdelen af den högre Axelen / den
% fremste eller högre Foten corresponderar, hwilket och i qvarta a: p:
% f:, är att observera / allenast att man med Ansichtet / i dett samma
% (mädan handen blifwer innan till wänder) seer bak, eller utom Klingan
% fram; Udden låter du i stöten någt siuncka / der med de under-blotten
% emot hans förfallning / blifwa bättre förswarade: Ditt högre Knää,
% måtte du wäl i stöten utsträckia och kröka / handen wäl fram för sig
% utfördt / och när stöten är giord / hurtigt och geswindt åter igen
% gå tillbaka. Med ligation skeer, det fördenskull / att man är bättre
% betäckt med hela kroppen till hans högre sijda. Och så när du emot de
% medla och under gvard / med udden upp / och korset sänckt / utan till
% öfwer swagan hans Klinga instöter.

thrusts, thus the same, but the upper part of the right Shoulder, the
foremost or right Foot corespond, which in fourth a: p: f:, is
observed, alone with the Face, in the same (while the Hand is turned
to the inside), look back, or outside the Blade forwards. The point
you let in the thrust sink slightly, wit the under-opening against his
fall, are better protected. Your right Knee, you must well in the
thrust extend and bend, the hand well ahead of you stretched, and when
the thrust is complete, quickly and with alacrity again return. With a
bind is done, for the sake, that one is better covering te whole body
to its right side. And when you against the middle and lower guard,
with the point upwards, and the hilt lowered, on the outside over the
weak, his Blade thrust in.

% item. när du emot de höga gvardier / kroppen tillförende skränker /
% och der på utan til öfwer Klingan med passaden ingår / läderar du
% Fienden med den samme uti tertia; Först utan till öfwer Klingan /
% när han han det sammastädes i en hög eller lägre gvard' / på din
% string' eller Fint per Tertiam med den Vantagio af Contrapostur
% blifwer giord / ligger stilla och låter dig de blott till stöten
% stå öpne / eller innan til caverar / der stöter du effter Fiendes
% gvard' och rörelse i en sådan proportion.

Item, when you against his right guard, angle your body, and then on
the outside over the Blade with a passage go in, you injure the Enemy
wit this in third. First outside over the BLade, when he in this in a
high or lower guard, against your bind or Feint in Third with the
Advantage of Counterpoise is done, you stay still and allow the
opening to the thust be open, or disengage to the inside, you thrust
after eth Enemy's guard and motion in such a proportion.

% Men du måste hos de utwärtes blotten öfwer klingan / hans högre Arm
% tillförende den distance, aff udden och hans kropp considerera / om
% hans motion aff hans Arm / genom hwilken han din Klinga i {\it
% Voltering} med qvarta kan parera eller försättia / är minde än din
% stöt / hwar effter du måste rätta dig.

But you must in the outside openings over the blade, his right Arm
addind to the distance, of the point and his body consider, if his
movement of his arm, through which he your Blade circle in fourth can
parru or emplace, is less than your thrust, in which you have to follow.

% Wore du honom utan til / emot en under gvard högt angången, och han
% derstädes giorde ett tempo med Klingan / så stöt i det samma (der
% med at undfly större {\it Motioner} i en jämbd linea) utan rörelse
% hans Klinga fort / och strin' honom med ligation för hans
% efterstöt. Emot hans {\it Muterande} i en öfwe eller medels {\it
% tertia}, kan du wäl först utan til med din Klinga a traverso öfwer
% hans, eller i dett ställe utur presenza gå undan hans udd, så kan
% hans {\it mutering} intet skada dig, och då med en hast utan til öfwer
% hans Klinga per teriam uphöja odden til hans högre Axel fortstöta:
% Wille han i dett, en qvarta under din Klinga voltera, så låt korset
% i din stöt tillijka med kroppen uti en hast nedsiunka så kan han
% intet bruka qvarta.

Were you him outside, against a low guard highly approaching, and he
then made a tempo with the Blade, then thrust at once (thus avoiding
larger Motions in an even line) without movement his BLade quickly,
and bind him with a bind for is afterthrust. Against his Change in an
upper or middle third, you can well first outside with your Blade
traverse over his, or in its stead out of presence avoid his point,
thus his mutation cannot hurt you, and then with haste outside over
his BLade in third lift the point to his right Shoulder thrust. If he
wants in this, a fourt under your Blade encircle, let your cross in
your thrust close to your bodylower, and he cannot use fourth.

% Denna försänkning med korset tienar och när Fiended söker att råka
% dig under tillijka med sig, på din utan eller innan stöt per Tertiam
% eller qvartam öfwer Klingar wånter[???]. Och om man giör den
% rörelsen uthi en empito eller behändig och fächta mothålning, skulle
% hans udd ned til jorden eller sijdan bortfalla, och stöten icke den
% mindre tillijka uti ett tempo med så ringa motion som tillförende
% kunna blifwa förrättadt, om elliest din wederpart geer dig utan till
% ett tempo, och söker att förhindra din {\t Cavation} med sitt
% korssänkande (doc utan kroppsens {\it abaissering}) så gack du med
% din klinga öfwer och stöt utan til (emedan han derstädes med sin är
% förswagat) i en jämbd linie till hans högra Axel.

This lowering of the cross serves also when the Enemy seeks to reach
you below, on your outside or inside thrust in Third or fourth over the
Blade [???]. And if this movement is done in an [empito???] or
[beh{\"a}ndig???] and fencing opposed[???], should his point towards
the ground or the side fall away, and the thrust no less be done in
one tempo with the smallest motion that could be done, if only your
opponent gives you a tempo on the outside, and tries to block your
Disengage with his cross-lowering (only without the lowering of the
body), you should go with your blade above and thrust on the outside
(since he is weaker there) in an even line to his right Shoulder.

% Wijdare blifwer och tertia med cavation utan till öfwer Klingan som
% tillförende är omrört, effter {\it proportion} och lägenheet sin
% Fiendes motion, eller emot hans swaga, med korset wäl betäckt,
% instöt; Hwilket skeer när han träder in mensura med eller utan
% cavation och söker med sin styrka din swaga.

Furher it is that third with a disengage outside over the Blade as
previously mentioned, after proportion and situation with the Enemy's
movement, or against his weak, with the cross well covered,
thrusted. Which should happen when he steps into measure with or
without disengage and seeks with his strong your weak.

% Item, när han på din inwertes Fint per qvartam med constrapostur
% till hans

Item, and when he on your inside Feint in fourth weith contraposition
to his
