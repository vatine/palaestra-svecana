\newpage

\scan{079}

\newpage

% Translation below
\exercise{How a hanging Second or Third are in thrusts done and used}
%\exercise{Huru förhängande Secunda eller Tertia blifwa i stöter giorde och brukade}
% Ligger din Advers' långt uthsträckt i Secund' Tert' eller qvart' med
% en jembd Klinga/ så gack med förskränkter Kropp per Secund. i en
% jembd linea innan till hans kårs / ungefär ett hallft qvarter der
% ifrå med högre fotens oplyfftning / och låsz som du der wille
% instöta/ gack altså i en hast igenom med en subtil rörelse af
% Klingan öfwer hans sinkos per Secund' eller prim. och stöt tillijka
% med ligation utom hans Klinga till hans underlijff: Kroppen blifwer
% wähl försänkter och gack med kårset wähl högt Fig. 8.
If your adversary is extended far in second, third or fourth with a
level blade, then go with hunched Body in Swecond in a level line
inside his cross, roughly a half quarter\sidenote{The quarter is a
``quarter aln'', so six inches. At the time of the original, the Swedish
inch was 24.74 mm, so a quarter is 148.44 mm, and a quarter is thus
roughly 74 mm, or just shy of 3 modern inches} away, with a raising of the
right foot, and pretend that is where you wnat to thrust, then go
rapidly through with a subtle movement of the Blade over his bend in
Second of first and thrust then with a bind outside his Blade to his
groin. Body well lowered and go with the cross well high. Fig 8.


\exercise{}
% Will han din Fint per qvart med förhängder Secund' tillijka i en
% omwridning ut' parera, i mening utan till under din klinga att
% instöta / så wexla utur qvart' i tert. och stöt henne utan till
% under hans klinga in / så att han råkar med sin swaga i din styrkia
% / då är honom hans kringwridning förhindrat / och hans klinga
% ligerat: Detta kan du och bruka när han stöter qvart' innan till och
% på dett sätt will parerar. Fig 8.
If he your Feint in fourth with hanging Secund further in a twist
outwards parry, meaning to thrust in outside under your blade, switch
out of fourth into third and thrust it outside under his blade in, so
taht he edns with his weak in your strong, then to him is his twisting
denied, and his blade bound. This you can also use when he thrusts
fourth inside and in this manner will parry. Fig. 8

\exercise{}
% Stringera honom innan till / går han i dett samma under din klinga
% igenom / och wille utan med förhängder tert' till din högre sijda
% instöta / så stöt med en hasst secund' i en linea utan till under
% hans klinga / för än han kommer under din / lijka som med en
% omwridning; I lijka motte när han på den inwertes finten per qvart
% wille giöra dett samma. Fig 8. Alltså och när han i ett underläger
% på din inwetz string', med en angulerat secunda till din inwertz
% kropp wille stöta.
Bind him on the inside, if he in that moment goes under you blade, and
thrusts outside with hanging thrird to your right side, then thrust
swiftly second in a line outside under his blade, before he gets under
yours, also with a twist; Similarly when he against the inside feint
in fourth will do the same. Fig. 8. Thus also when he from a low
position on your inwards bind, with an angled second to your inside
body is thrusting.


\exercise{}
% Derföre string. honom innan till / och giör honom derstädes finten
% per qvartam; wille han då med secunda utan till under din klinga
% instöta / så parera honom hans stöt / med din hela styrkia till din
% högra / lijka som med en omwridning och stöt i ett arbete utan till
% per tertiam öfwer hans klinga till hans högre Axel.
Therefore bind him on the inside, and then do him the feint in fourth;
if he then with second on the outside under your blade thrusts. then
parry his thrust with your whole strong to your right, also with a
twistand thrust him in one work outside in thrid over his blade to his
right Shoulder.
