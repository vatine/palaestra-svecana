\newpage
\chap{Dictionary}

Here follows an approximate dictionary from original term to translation used. This is not 100\% consistent throughout, as the translation work has taken place over an extended time.

\begin{description}
\item{abaissera} (verb) lower
\item{alpari} (adv) nominal
\item{battut} (noun) beat
\item{cavera} (verb) Cavazione, disengage
\item{chiamat, chiamata, chiamater} (noun) Invitation, invite
\item{chiamat} (verb) invite
\item{contrapostur} (noun) counterpose, counterpoise
\item{coopert} (adj) occupied
\item{disordonera} (verb) disorganise
\item{empito} (noun) surge
\item{enga} (adj) narrow
\item{falcera} (verb) the corresponding noun is ``falcad/falkad'', which is defined as a ``hastig kurbett'', which would be a ``quick courbette''. I take this to mean a quick rotation of the blade, with the hand remaining in roughly the same position.
\item{ligera} (verb) bind, possibly more so than ``stringera''
\item{mutatis mutandis} (adj) Roughly ``with appropriate changes''
\item{opprimera} (verb) press down, press together
\item{presenza} (noun) presence
\item{ringande} (noun) hand-to-hand combat, wrestling
\item{scurtzo} (adj) Not entirely sure, this could be ``short''.
\item{sinkos} (adj) bent, slow, side
\item{stringera} (verb) bind
\item{string} (noun) bind
\item{skränkt} (adj) held at an angle, angled
\item{tentera} (verb) Try, attempt
\item{trattenera} (verb) Keep away
\item{voltera} (verb) Circle around
\end{description}
