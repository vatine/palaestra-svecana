\newpage

\scan{098}

\newpage

% Translation below

% \exercise{Öfwerkastning. Eller på hwad sätt man skall slengia och hafwa sin Klinga öfwer sin Fiendes.}

\exercise{Overthrowing. Or in what manner one shall throw and have one's Blade over that of one's Enemy}

% Lägg dig med jembd Klinga för din Fiende / så att du förer och
% behåller din hand och werja frij / der nu din Fiende skulle söka din
% Klinga innan till att string' / så gif gran acht uppå / i dett han
% träder i mens', att du intet låter röra din klinga / utan går ifrån
% din högra öfwer hans Klinga / och kastar qvart' utan till öfwer hans
% högra Arm till hans högre bröst: Fig 19.

Lie with an even blade before your Enemy, so that you move and keep
your hand and rapier free, when now your Enemy seeks your Blade on the
inside for a bind, then pay close attention to, that when he steps
into measure, that you do not let him touch your Blade, but go from
your right over his Blade, and throw fourth on the outside over his
right Arm to his right breast. Fig 19.

\exercise{}

% du kant och när din Fiende har string' din Klinga innan till / kasta
% öfwer hans Klinga angulerat qvarta till hans högre bröst; Men der
% han samma stöt till sin högra utparerar, går du uti en motion igenom
% / och stöter Qvarta innan til hans högre Bröst.

You can also when your Enemy has bound your Blade on the inside /
throw over his Blade angled in fourth to his right breast. But when he
in the same thrust to his right parries, you go in a single motion
through, and thrust Fourth to his right Breast.

\exercise{}

% Alltså och när du åft string' utan till / kastar du af frij stycken
% öfwer hans Klinga / i dett du går ifrån din wenstra till din högra /
% och stöter så qvarta innan till åht hans högre Arm; Men der han /
% skulle parera din stöt något till sin wenstra sinkos / så cederar du
% i dett samma / och låter din udd försiunka / stötandes qvarta under
% hans klinga till hans underlijff. Du kant och med din Klinga fara af
% och till i högden som Fig 18. utwisar / der igenom gör du din Fiende
% des säkrare / emedan din hand är altid i motu. Så kan han intet så
% wel judicera tempo när du wilt stöta / när du då har nått Mens', så
% kasta med en hast angulerat Qvart' utan till öfwer hans högra
% Arm. Fig 19. Du kant och när du förer din Werja i flychten som Fig
% 18. wisar. Giöra honom med udden ett alwarsam mijn till hans
% Ansichte / så lär han som ingen twifwel är / taga med sin styrka
% till sin högra (medan man är mest ömmer och redder om denna dehl)
% gack så med en hast med en god Avancement under hans Klinga / och
% stöt qvarta under hans högra Arm / när denna stöten blifwer wäl
% giord / står han intet at parera.

Thus when you bind from the outside, you throw free pieces over his
Blade, in that you go from your left to your right, and thrust fourth
inside to his right Arm. But were he to parry your thrust somewhat to
his lower left, you retreat instantly, and let your point sink,
thrusting fourth under his blade to his groin. You can also with your
blade go up and down in height as Fig 18. shows, through this you make
your Enemy safer\sidenote{This seems wrong, based on what follows, but
as far as I can make the original text out, that's what it says},
since your hand is always in motion. Thus he cannot judge that well in
what tempo you seek to thrust. When you then reach Measure, thrust
with alacrity angled Fourth outside over his right Arm. Fig. 19. You
can also as you are moving your Rapier in flight as
Fig. 18. shows. Make him with the point a serious grin to his Face, he
will as there is no doubt, take with his strong to his right (where
one is more tender and care more for this part) go with haste with a
good Advance under his Blade, and thrust fourth under his right Arm,
when this thrust is done well, he is unable to parry.

\exercise{}

% Du kan och först giöra en half öfwer-kastning innan till hwilken
% blifwer således giord såsom en half stöt; Tager han med sin Klinga
% til sin wenstra / så gack med en hast ifrån din högra öfwer hans
% Klinga / och stöt qvarta utan till öfwer hans högre Arm. Fig. 19.

You can also do a half throw over on the inside to which is thus made
like half a thrust. If he takes with his Blade to his left, then go
with haste from your right over his Blade, and thrust fourth outside
over his right Arm. Fig. 19.

\exercise{}

% Såldeds och när din Fiende har string. dig innan till / går du öfwer
% hans Klinga / och giör honom en half stöt / parerar han med sin
% KLinga till sin högra sinkos / så gack i dett samma ifrån din
% wenstra till din högra öfwer hans Klinga och stöt qvarta innan till
% hans högra bröst / med insteg af högre foten: Men der han låge
% stilla

Thus when your Enemy has a bind on you inside, you go over his Blade
and do him a half thrust, if he parries with Blade down and to his
right, then go from your left to your right over his Blade and thrust
Fourth inside to his right breast, with an instep of the right
foor. But were he to lay still
