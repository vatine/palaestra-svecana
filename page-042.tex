\newpage

\scan{042}

\newpage

% Translation below
% rörelse är så långsam; En om du i första motu i hans Fint stötte/ i
% dett han retirerar sin Klinga / förmenades henne för din parade att
% Salvera) lär du redan råka honom / förr än han har kommit att
% Slancera sin stöt.

movement is so slow. Even if you in the first move in his feint
thrusted, in that he retreats his Blade (intending it to Save against
your parry) you should alredy reach him, before he has
Counter-thrusted.

% Merck här i gemen / emot alla Finter, när du säkert i dem stöta
% wilt / att du tillförende altijd med Contrapostur befinner dig nähr
% wijd hans Klinga . i dett att du will stöta| icke först hans Fint
% med fara söka måste / och om du med Contrapostur, din Fiende redan i
% Larga mensura med fötterne och Klingan ferm, han wille då Fintera
% emot dig / med willia att wara din Klinga frij / kan du gå med din
% styrkia i hans swaga / och uti samme tempo stöta in på honom / och
% lär han (i dett han gick fram för sig) oplyfftandes foten / intet
% kunna parera, eller bryta mensuram, emädan han uhti ett tempo intet
% kan gå fram och tillbaka: Här jämpte när han låter wara di Klinga
% frij/ och du skulle parera effter hans Fint / kan han då intet råka
% dig / särdeles när du åst ferm med fötterne / kan du (när han på
% giorder Fint) stöta wilt / bryta mens' honom derigenom disordinera,
% och (förr än han har remiterat sig) träffa.

Note here that in general, against all Feints, when you safely thrust
into them, that you always are in Contrapostur close to his blade, in
that you want to thrust, not dangerously seek his Feint, and if you
with Contrapostur, your Enemy already in Wide measure, with firm feet
and Blade, he wants to Feint against you, wanting to be free of your
blade, you can go with your strong in his weak, and in the same tempo
thrust in on him, and he will likely be (in that he was moving forwards)
lifting is foot, not being able to parry, or break measure, since he
within a tempo cannot go forwards and back. In this when he lets your
Blade stay free, and you were to parry after his Feint, he can then
not reach you, especially when you stay firm on your feet, you can
(when he completes the Feint) thrust, break measure and confuse him
and (before he has regained composure) hit.

% Finterar han utan till effter ditt Ansichte/ så gif honom med en
% retirade något litet tempo till att passera, och om han går fort /
% stöter du i det samma qvart; lijka som han wore wreden innan till
% hoos hans Klinga / med uddens försänkning till hans underlijff. Du
% kan och i förste tempo med secunda stöte under hans Klinga / hade du
% detta försummat / och låtit honom passera, måste du has stöt utan
% till igenom en ligation med Klingans tilldragande och kropsens
% skränkning parera, och i dett samma resolvera dig till hans kors: Du
% kant och emot denne Finten med en inwertes Contrafint gå per
% qvartam.

If he Feints on the outside to your Face, give him with a retreat a
small tempo to pass, and if he is quick, you thrust in that moment in
fourth, just as if he was angry on the inside of his Blade, with the
point lowered to his groin. You can also in the first tempo thust with
second under his Blade, if you have neglected this, and allowed him to
pass, you must his thrust on the outside with a bind with the pulling
back of the Blade and the twisting of the body parry, and in the same
resolve yourself to his cross. You can also against this Feint go with an
inside Counterfeint in fourth.

% På den inwertes Finten per qvartam effter Ansichtet / kan du med en
% lijten retirade gifwa honom blott / och när han der på will passera,
% wänder du handen i qvarta med försänkter udd voltera, eller retirera
% dig / så att du kommer med din udd uti ett medel-läger; Passerar han
% i dett samma fort/ kan du bruka samme volt.

On the inside Feint in fourth against the Face, you can with a small
retreat give him an opening, and when he after that wants to pass, you
turn your hand into fourth, with lowered point spin\sidenote{The
original uses ``voltera'', which is the verb form of volte, later text strongly indicates that this is simply ``turn the body in a full circle, quickly''}, or retreat, so
that you end up with your point in a middle position. If he passes in
the same quickly, you can use the same circle.

% Finterar han innan till per tertiam utan jern / kan du med
% försänkter udd stöta qvarta' till hans högre hyfft / så blifwer
% honom Cavation till utwärtes stöten förbuden; Eller kommer han intet
% fullkommeligen i mens' / så avancera i hans Fint med halfwa qvarta,
% och i dett samma han till stöten igenom går / förändra i ett tempo
% din effect / och koom honom utan till med stöten per Tertiam
% emot. Item parera med Klingan till en förförning / låt i dett samma
% din udd något siunka / så har du henne till cavation sedan des
% frijare / stöter han wijd dett samma / så gå igenom under hans
% Klinga

If he feints on the inside in third without steel, you can thrust with
lowered point in fourth to his right hip, thus he is denied a
Disengage to outside thrust. Or is he is not fully in measure,
advance his Feint with half fourth, and in the moment he goes to
thrust, change in one tempo your effect, and come at him on the
outside with a thrust in third. Parry with the Blade to a translation,
let in the same moment your point to sink slightly, you have it more
free for a disengage, if he thrusts in this, go through under his Blade.
