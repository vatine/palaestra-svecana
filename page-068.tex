\newpage

\scan{068}

\newpage

% Translation below
\exercise{How to thrust on a Disengage}
%   Huruledes man skall stöta på en Cavation
% String' din Fiende utan till / Caverar han / så stöt / för än han
% har ändat sin Cavation innan till med Qvart a. p. fermo med et
% insteg af högre foten till hans högre bröst (Fig 2) Caverar han i
% dett samma du stöter Qvart. innan till / så förwänd din hand utur
% Qvarta i Tert' och stöt henne utan til övfwer hans Arm till hans
% Bröst (Fig. 4)
\sidenote{This is the first of two pages of translation for the same source page.}Bind your Enemy on the outside, if he Disengages, then thrust, before
he has finished the Disengage, on the inside with a. p. fermo in
Fourth with a step-in with the right foot to hist right chest (Fig. 2)
If he Disengages in the same that you thrust in Fourth on the inside,
then turn your hand out of Fourth into Third and thrust the blade on
the outside over his arm to his Chest (Fig. 4)

\exercise{How Feints are done}
%    Huru Finter blifwa giorde
% Ligger Adversarius i Tertia och har upphögt udden / så
% String'. honom utan till / ligger han stilla / så giör honom med en
% hast en fint innan till per Tertiam uti en jämbd linie / utan jern /
% ungefär en Qvarter fram för hans kors till hans medel:liff / och med
% foten Sosp. in Aria. Parerar han till sin wenstra / så Cavera / och
% stöt utan till hans högra Axel per Tertiam a. p. fermo; Caverar han
% under din stöt så förwäxla utur Tert. i Qvart'. och stöt honom innan
% till hans Kropp: Men Caverar och parerar han din stöt till sin
% wenstra så pasera och stöt per Tertiam wälbetächt / till hans högre
% sijda eller under hans Arm.
If your Adversary is in Third and has an elevated point, then Bind him
on the outside, if he is at rest, do him with haste a feint on the
inside in Third on an even line, without iron, about a Quarter in
front of his cross to his middle body\sidenote{This would be somewhere
just above the navel}, and wit hthe foot suspended in the air. If he
parries to his Left, then Disengage, and thrust on the outside to his
right Shoulder in Third a pie fermo. If he Disengages under your
thrust, change out of Third into Fourth and thust him on the inside to
his body. But if he Disengages and parries your thrust to his left,
then pass and thrust in Third well-covered, to his right side or under
his Arm.

\exercise{}
% String' honom innan till / ligger han då stilla så Cavera och giör
% honom i hast en fint med half Qvart. till hans högre Axel / ungefär
% med ett Quarter jern / när wijd hans Kors; Parerar han till sin
% högre / så stöt med Cavation innan till hans högre Axel a. p. fermo;
% så att / när Cavation är ändat / då moste stöten hafwa träffat / och
% förblifwer Cavation med Klingan emot hans swaga så att udden allena
% giör effected. (Fig 2)
Bind him on the inside / if he then is still then Disengage and do him
in a haste a feint wit hhalf Fourth. to his right Shoulder, roughly
with a Quarter iron, near to his Cross. If he parries to his right,
thrust with a Disengage to theinside his right Shoulder a. p. fermo,
so that when the Disenage is competed, the thrust must have landed,
and remain Disengage wit the BLade against his weak so taht the point
alone does the effect. (Fig 2.)

\exercise{How these both Feints are doubled}
%    Hurulededs desze begge Finter blifwa doublerade.
% Ligger Advers- Tert. något långt uphögd / och med udden inwers
% wender / så String' honom utan till / och Cavera med en hast innan
% till och giör honom en fint med en battute af foten / gak åter
% igenom men en hast / och giör honom den finten utan till med foten
% sospeso in aria; parerar han till sin högre så Cavera och stöt per
% Qvartam a. p. fermo (Fig 2).
If your Aversary is positioned somewhat long and elevated, and with
the point inwards turned, then Bind hom on the outside, and Disengage
with a haste to the inside and do him a feint with a stomp of the
foot, then go through with a haste and do him that feint on the
outside with the foot suspended in the air, if he parries to the right
then Disengage and thust in Fourth a. p. fermo (Fig 2).

\exercise{}
% Ligger din Advers. långt och med något ophåfver udd / så gå honom an
% per Tertiam med udden i prospective hans kors intent mycket diupt,
% träd altså med wenstre foten i qvadro / och drag dig med Kroppen
% utur presens hans udd / med Klingan blif i en jembd linie under hans
% Klinga och holt den högre foten i samma linea sospeso in aria går
% han i det samma med sin Klinga prospect' din Kropp effter / så sätt
% med en hast den högre i den förrige linien för sig neder / och
% wijdare i Mensura, giör altså honom Finten innan till per Qvartam
% och gif wäl acht uppå huru han parerar, parerar han intet / så stöt
% per Qvartam derstädes med passaden af wenstre skenkeln Fig 9.
If your Adversary is extended and with a slightly elevated point, then
approach him in Third with the point seeking his cross, not very deep,
step thus with teh left foot in fourth, and pull yourself with the Body
out of presence of his point, remain with the Blade in an even line
under his Blade and hold the right foot in the same line suspended in
the air, if he at that time with his Blade seeks your Body, put with
haste the right in the previous line dow and further in Measure, do
thus him the Feint on the inside in Fourth and pay wel attention to
how he parries, if he does not parry, then thrust in Fourth with the
passing of the left shin Fig 9.

\newpage

\scan{068}

\newpage

\exercise{}
% Merk i gemen hos alla Finter, att dhe blifwa säkrast och lyckas bäst
% / när din advers' är i motu / eller gifwer dig antingen med foten /
% Kroppen eller Klingan ett tempo och du i dett samma giör en Fint, då
% låter han den intet så lätt kunna judicera.
\sidenote{This is the second of two pages of translation for the same source page.}Note that in common with all Feints, that they are safest and succeed
best, when your Adversary is in movement, or gives you either with the
foot, the Body, or the Blade a tempo and you in that instant doa
Feint, then he will not easily be able to judge.

\exercise{How to Disengage and Counter-Disengage}
%   Huru man skall Cavera och Contra-Cavera
% Lägg digh uti ett långt läger / med udden något inwertes wändt /
% wille han då din Klinga utan till med ett instegh String', så gack i
% en hast med cavation och stöt qvart. innan till a. p fermo till han
% högre Axel (Fig 2)
Be in a long position, with the point somewhat inwards turned, if he
then tries to bind your Blade on the outside with an in-step, then go
in a haste with a disengage and thrust fourth on the inside
a. p. fermo to his right shoulder. (Fig 2)

\exercise{}
% Träder han i sin String. med foten intet fram för sig/ så giör
% honom i hans String innan till en fint och gif acht uppå hans
% parade.
If he treads not in his Bind with the foot in front of him, do him in
his Bind a feint on the inside and take note of his parry.

\exercise{}
% String'. honom innan till / caverar han och wille dig åter utan till
% string'. med ett insteg/ så contracavera och stöt i dett samma
% contra cavation innan till per qvartam a. p. fermo (fig 2)
Bind him on the inside, if he disengages and tries to bind you again on the outside with a step in, counter-disengage and thrust in the same counter-disengage in the fourth a. p. fermo (fig 2)

\exercise{}
% Stringera honom utan till / caverar han och will dig åter med ett
% insteg innan till string'. eller med sin styrka i din swaga fintera
% eller stöta/ så contracavera och stöt utan till med contra-cavation
% a. p. f. träder han med sin cavation intet fram för sig / så giör
% honom en fint i contra-cavation utan till / och stöt qvart. innan
% till (Fig. 2.)
Bind him on the outside, if he disengages and tries again with a step
in to bind on the inside, or with his strong in your weak feints or
thrusts, counter-disengage and thrust on the ouside with
counetr-disengage a. p. f. if he does not advance with his disengage,
do him a feint in counter-disengage on the outside, and thrust fourth
on the inside. (Fig. 2.)
