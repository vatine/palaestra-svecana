\newpage

\scan{096}

\newpage

% Translation below

% gran acht / i dett han giör Cavation, att du förwender utur qvart' i
% tert' och stöter så till hans bröst: Men der han gifwer sig i sin
% högra sijda öppen / stöter du Flanconnade under hans högre Arm som
% Fig. 16. utwijsar.

great attention, in that he disengages, that you turn out of fourth
into third and then thrust to his chest. But where he gives an opening
on his right side, you thrust Flanconade\sidenote{A move that starts
withthe blade on the outside, turning over and inside, pushing the
blade aside, ending in a downwards thrust} under his right Arm, as
Fig. 16. demonstrates.

\exercise{}

% Wijdare med Ligering. Ligger din Fiende ännu med jembd Klinga för
% dig / så string' honom utan till / Ligera honom hans Klinga så att
% du kommer åter med din Klinga utan till hans swaga; Ligger han der
% på stilla / så stöt tert' ända fram;
Further in binding. If your Enemy is still with a flat blade in front
of you, then bind him on the outside. Bind his blade so that you come
again with your blade outside by his weak. If he there lies still,
then thrust third all the way.
%                                       Men gieck han i dett samma
% under din Klinga igenom / så förwändt utur tert' i qvarta, och stöt
% till hans bröst inwersz; Farer han med sin Klinga bak öfwer sig /
% och kommer med sin halfwa sryrkia i din halfwa styrkia / motte du
% intet cavera utan du förwänder din hand utur tert' i prima, och
% stöter denne öfwer hans Klinga så lärer stöten råka in i hans hals
% Fig 6. Tager han till sin högre sinkos / caverar du i samme tempo
% och stöter qvarta innan till hans bröst.
But if he in that instant goes under and through yiour Blade, then
turn from third into fourth and thrust to his chest's inside. If he
brings his Blade behind and over, and comes with his half string in
your half strong, you must not disengage but turn your hand out of
third into first, and thust this over his Blade, then the thrust will
hit his throat. Fig. 6. If he takes to his right lower, disengage in
the same tempo and thrust fourth inside to his chest.
%                                          Merk här hos åtskillnaden
% emellan circulering och ligering; Nähr du wilt circulera motte din
% Klinga wara af din Fiende string'. Men när du wilt ligera så moste du
% hafwa string', och när desze Lectioner blifwa wähl giorde har
% Fienden stor möda att befrija sigh.
Note here the difference between circling and binding. When you want
to circle your blade must be bound by your opponent. But when you want
to bind, you must be in control and when these lessions are well done
your Enemy will have great difficulty to get free.

% \exercise{Huru man emot Chiamater sig förhålla skall.}
\exercise{How to engage with invitations}
% Ligger din Advers' högt och retirerar sig i ett medel läger /
% gifwandes sig innan till blott / så string' honom innan till / träd
% i dett samma med högra foten effter / så snart han wille då innan
% till stöta / så stöt Qvarta med dett falska steget tillijka med
% honom till hans inwersz kropp. Fig. 11.

If your Adversary is high and retreats to a middle distance, giving an
opening on the inside, then bind him on the inside, step in with your
right foot at the same time, as soon as he seeks to thrust inside,
thrust Fourth with the false step at the same time inside to his
body. Fig. 11.

\exercise {}
 % Wore du i Mens' string', eller trädde in och gick med sin udd till
 % sin högra sinkos / och gifwo dig innan till med wilia blott / så
 % stöt i dett samma qvarta med dett falska steget / till hans inwersz
 % kropp: Men låter han udden siunka till sin högra / så blif med ditt
 % kors högt / och giör honom en fint innan till / wille han i dett
 % samma Voltera så giör en contra-volt.

If you are binding Measure, or stepped in and went wit his point to
his lower right, and gives you on the inside willfully an opening,
then thrust instantly in Fourth with the false step, inside to his
body. But if he lets the point sink to his right, keep your crossguard
high, and do him a feint inside, and if he Voltes then do a
counter-volte.

\exercise {}

% Ett annat sätt / emot de uhtwertsz chiamater. Woro du innan till i
% Mens' han då caverade, och utan till wille giöra en chiamat förutan
% retirering. i tankar när du skulle utan till per tertiam stöta / han
% då wille voltera qvart under din klinga så stöt

Another way, agains outside invites. If you were inside in measure
when he disengaged, and intended to invite on the outside without
retreat, thinking to that you were to thurst third on the outside, he
were to volte fourth under your blade then thrust
