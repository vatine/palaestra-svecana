\newpage

\scan{012}

\newpage

% Translation below

\chap{What Measures are}
 There are two measures\sidenote{Original says ``Mensura'', but
 ``Measure'' may be easier to read},Wide and Narrow.Wide is when one with
 a thrust and the right foot following can reach one's Enemy.

To safely execute this measure, it is necessary when one with a steady and
ready arm only touch the foremost part so that I win Contrapostur or
a binding\sidenote{The original uses ``stringera''}.

It is called Contrapostur when your adversary has offended against a
Wide Measure. First ensure that your blade covers your Enemy's point
on your body, then you can occupy his weak with your strong.

% Here is a transcription
% hwad stringera anbelangar / kommer dett Contrapostur i månge delar
% övereens, och hafwer lika regles of egenskaper / undantagandes i de
% händelser / att offta den linien som kommer af des adversari udd som
% går till din kropp / at hans klinga aff din klinga intent bllifver
% rörd / på dett i medlertijd kan din fiende intet gå med sin udd till
% din kropp / med mindere han icke råkar med sin svaga i din styrkia
%
% hwarföre af fölliande orsaker skall man inte anrora sin Fiendes
% klinga uti stringerande, utan du förer din Klinga öfver ans iospeso
% in aria, eller såsom swäfwande. hwad beträffar den förste fördehl
% uti strngrande, skeer eller består icke efter sombliges wana och
% mening / att tvinga och påtrycka deras Fiendes Klinga neder mot
% Jorden / hwilka förmena der igenom att winna sin Wederparts Klinga
% icke förnekandes uti hwad fara de sielfve äro stadte i dett de alt
% förmycke förlåta sig uppå sin Fiendes Klinga; Slår den som så hårdt
% blifwer molesterat och nederpressat / lårer alltd wara omtänksam /
% huru han sig ifrå så hård string' kan aflossa och befria / khilkwt
% låter sig giöra med Finter och Scheamater elliest om den andre
% behändigt Caverade, eller låte sin udd fi???a försummade du igenom
% en Cadur din klinga de tempo; förmår han då hålla emot din Klinga
% med sin Klinga / händer sig offta att man då kommer utur fechtande
% till ringande. Skulle ditt tempo på det sättet

As far as Binding is concerned, it is in many ways similar to Contrapostur
and have similar rules and characteristics, excepting those occasions
when the line of your Adversary's point would not let his blade be
touched by yours, but also that his point cannot touch your body
without letting his Weak into your Strong.

Thus for the following reasons should you not touch your Enemy's blade
and bind it, but you move your Blade over his hanging in the
air\sidenote{``sospeso in aria''}, or hovering. Concerning the first
advantage of binding, is not to press your Enemy's blade
towards the Earth, those who say this are in denial on how much of a
danger they are in by relying too much on their Enemy's blade. 

It should always be a concern how he from a hard bind can loosen or
free, which is accomplished with Feints or Schemes, or if the other
handily disengaged or lets his point find you unaware through [Cadur]
your blade [de tempo], which would allow him to hold against your
blade with his blade, this happens frequently when coming out of Fencing into
wrestling.

Should your tempo in this way
