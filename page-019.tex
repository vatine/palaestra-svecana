\newpage

\scan{019}

\newpage

% Translation below
% att du kan dig Avicinera, då måtte du märkia grant / om han rört sig
% med list / hwarigenom han will reta dig till att stöta / på hwilket
% tillfälle / har du intet behof att stöta / och der du äntelogen aå
% wille / måste du föra din klinga på Fintwijs / och med dett samma gå
% till de blott / och i det samma han will taga Contratempo, måtte der
% jämpte för"andra din rörelse afwijkandes din kropp ifrån hans udd/
% och går sedan till dett andre blott om han geer öfwer contratempo,
% whilket man elijest ( der ingen wiszhet är att råka ) altijd måste
% observera, så har du brukat list emot list/ s/asom denna wettenskap
% der uti mestendehls består/ att man sin Fiende med behändigheet weet
% förföra och bedraga. Du kan och emot dhe chiamate i larga mensura
% stringera, och der hos taga i acht/ såson i Fine Cap 3. om mensura
% larga är förhandlat.

that you can [Avicinera ???], then you must notice [ grant ???] if he
has moved cleverly through which he teases you to thrust in a moment
when you have no need to thrust / and that when you finally want to
must move your blade deceptively and in that same go to the opening
and he will in that moment take Contratempo, you must then change your
movement avoiding his point and then go to the other opening if he
gives contratempo, which one must otherwise (when no certainty is to be
has) always observe, then you have used cleverness against cleverness,
such as this science mostly is composed of. You can also against these
invitations in wide measure bind, and take them into account, as is
described in Fine CHapter 3, about the wide measure is described.

% Men när du befinner dig i mens' stricta, kan du på det ringeste tempo
% som din Fiende geer dig/ utan mensurens brytning med blotta kropsens
% öfverbönning råka; En om i det samma/ då han gaf tempo, skulle hafwa
% satt sin Foot tillbakas / hade han fått tijd nog att parera och
% stöta / emedan han war den förste som rörde sig/ då kunde han och
% wara den förste som förändrade sin effect / hwilket han intet hade
% kunnat giordt/ när han hade stått stilla med föttren / uan då först
% welat rört dem när hans Fiende war i motu; En förr än han då hade
% brutit mensuram eller kunnat parera, har han blifwit stötter. Är
% altså/ utan retirering i denna mensur, farligit för den so först rör
% sig.

But when you find yourself in narrow measure, you can on the slightest
temp that your Enemy gives you without breaking measure reach merely
by bending your body.But if in the moment that he gave tempo, should
have brought his foot back, he would have had time to parry and
thrust, since he was the first to move, then he could also be the
fisrt to change his effect, which he would have been unable to do, if
he had kept his feet still, then only wnating to move them as his
Enemy opposes. Thus before he ahd broken measure or been able to
parry, he would have a thrust land. It is thus dangerous without
retreat in this measure, dangerous to be the first to move.

% Efter såsom den wijda mensura behöfwer patience, och den enga
% geswindheet/ derföre skall man på en cavation, eller på en annan
% rörelse/ när man will stöta sin Fiende/ intet wänta så länge till
% des han har fullbordat sin rörelse/ utan i dett samma han begynner
% sin motion, stöta i samma blott/ så lärer du i hans cavation op
% primera och laedera: At man och i denna Mensur, uthan någon
% affwäntning i tempo, genom den fördeel med contrapostur, r8/aka kan;
% hwar om är tillförende i Mensura stricta förklarat. Och skall man i
% detta fall/ emot sin Fiende/ lijta såsom en Katt när han är
% begrepen at attackera något så småningom smyger sig; Alltså
% oförmodligen winner och mens', eller elliest något fördeelachtigt
% hwarigenom ( förr än hans Fiende blifwer det warse ) kan warda stött
% och attrapperat. På annat sätt/ kan man moot sin Fiendes
% stillaliggning intet wäl procedera.

Since the wide measure requires patience and natural swiftness / thus
on a disengage, or any other movement, when you want to thrust your
opponent, not wait until he has completed his movement, but as he
starts his motion, thrust in that moment / so you will be [primera ???
and [laedera???] in his bet. You can in this measure, without waiting
for tempo, you can reach advantageously with contrapostur, which is
explained in Narrow Measure. And you should wait, like a Cat ready to
attack something sneaking by, attack your Enemy. Thus you will likely
win and have measure, or at least some advantage (before your Enemy is
aware) that can be used to attack and thrust. You cannot otherwise
proceed well against your Enemy's stillness.

\chap{Disengages, counter-disengages, Ricavation, Mezza-cavation; Committere
Dispada, what these are, and thus how they should be used}

% När din Finde will stringera eller ta din Klinga / och du i dett
% samma ?? från den ene sijdan till den andra ( för 
% än han rör din Klinga) går är en cavation i rättan tijd giordt / och
% blifwer fördenskull kallat cavation di Tempo

When your Enemy wants to bind or take your Blade, and you in that
moment [???] from one side to the other (before he touches your blade)
go, is a disengage done in correct time, and is thus called cavation di Tempo.